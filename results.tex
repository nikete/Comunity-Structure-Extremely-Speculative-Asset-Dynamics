All reported modelsa and results are using only $60\%$ of the data, the other $40\%$ is being held out. The researchers are blidning themselves to it, until the paper is ready for publication, so as to have a measure of true out of sample fit. 
We present results for $5$ models fitted with our methodology to each of our two outcome measures in both USD and BTC terms, 4 total outcome vairables.

Tables 1 and 2 report the coefficient estimates for standardized variables for the two outcome variables where we have the most substantial explanatory power beyond coin vintage: severity in USD and magnitude in BTC. While normally comparing R squared accross moels is not meaningful, the combination of having the same set of explanatory variables, while the depdent variables all have the same mean and standard eviations, does make them into a useful sumarries of model fit in this particular case. 
While the results for severity are not mcuh affected, the magnitude measure is less predictable in USD than in BTC.
We do not include the tables for severity in BTC and magnitude in USD to meet page limit restrictions.
There are no subsntantial changes beyond fewer variables being selected by the elasticnet for mangnitude BTC, which is consistent with its lower predictability.

Our User model considers how long a user has been active, and how many users have replied to them and he has replied to. 
Days since first post is negative, while the number of subject the user posts accross is positive. This could be interpreted as earlier members of the comunity being more earnest, and thus even when they start new blockhains theseare les likely to be the most aggresive pump and dump schemes. 

Our Network model encodes directed edges when a user interacts with another, and proceeds to estimate closeness centrality and clustering measures based on this. 
Our Weighted model uses the intensity of interactions to create a weighted network and construct a similar set of measures to the Network model.
The results for severity are very similar accross both USD and BTC, and largely show that closeness centrality and satoshi distance are have explatory power beyond time. Both  are positively associated with a bubbles severity.
While closeness centrality hasa negative coefficient in the magnitude table of results, this is not comparable accross models, since it is in models that control the clustering coefficients and users pagerank in the network, measures that donot make it through the Elastic Net in the severity model. 
Our Satoshi model considers the distance and pagerank relative to the user satoshi.
The same sign reversal between magnitude and severity is present in Satoshi distance, with bubbles of larger magnitude being further away from satoshi, and the more severe ones closer.
This is a counter-intutive finding given if ones priors are that those closest to satoshi have higher social capital in the network and are less likely to be running pump and dump schemes, and that they are more likely to attract investment. 

Our All model contains all previous mentioned variables, there is extreme collinearity in this model and thus it's estimated coefficients and standard errors should be interpreted with extreme caution. 
We also considered if a coin is non-trivial and it's interaction with th network and satoshi metrics. It did not consistently improve the model fit, and is thus excluded from reporting. 











