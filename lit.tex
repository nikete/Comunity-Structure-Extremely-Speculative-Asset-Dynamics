\section{Lit}

altcoins lecture and slides, cite course materials.
%https://docs.google.com/presentation/d/159-i63Fk6O7SOYfpA6mlzqTSlrRc7nBlOSS11q0robQ/edit#slide=id.g18f88b7d4_05



Agents structural strength within the discourse surrounding cryptographic currencies is particularly important, as these are almost entirely performative; the initial marketplace adoption of bitcoin is 
crash of  %87 http://www.iijournals.com/doi/abs/10.3905/jpm.1989.409242?journalCode=jpm 
“rapid rise of options” with the rise of the theory to price those options, in having currencies with zero entire costs
MacKenzie, Donald. "An Engine, Not a Camera."
While states can create the demand required for a currency system to run by compelling tax payment in it (for a recent example), non state sponsored currencies must find some other ways of creating demand.
The initial market for which bitcoin has been used (prices denominated in it, transactions only in it) where drug sales. citation.
New currencies have thus been floated with every single drug name possible. Many chains can claim to the same claim, so exchanges (since speculation is the only possible use of almost all of them) become de-facto stabilizers of who has a minimally viable claim. 





"We develop a series of cross-sectional regression specifications to forecast skewness in the daily returns of individual stocks. Negative skewness is most pronounced in stocks that have experienced (1) an increase in trading volume relative to trend over the prior six months, consistent with the model of Hong and Stein (NBER Working Paper, 1999), and (2) positive returns over the prior 36 months, which fits with a number of theories, most notably Blanchard and Watson's (Crises in Economic and Financial Structure. Lexington Books, Lexington, MA, 1982, pp. 295–315) rendition of stock-price bubbles. Analogous results also obtain when we attempt to forecast the skewness of the aggregate stock market, though our statistical power in this case is limited."



The only time a tulip touched econometrica pages:
Tâtonnement stability of infinite horizon models with saddle-point instability
WP Heller - Econometrica: Journal of the Econometric Society, 1975 - JSTOR
... horizon case. 5 As Samuelson [17, p. 981] notes, ". . . history tells us that all tulip
manias have ended in finite time.... Every bubble is someday pricked." Knowing this,
people start to sell when prices reach ridiculous heights. But at ... 




Bubble forecasting in economics and finance has used features of the time series (TODO cites)




Network Diversity and Economic Development
http://www.sciencemag.org/content/328/5981/1029.full



"Hence, highly clustered, or insular, social ties are predicted to limit access to social and economic prospects from outside the social group, whereas heterogeneous social ties may generate these opportunities from a range of diverse contacts (1, 2)."

"Although both social and spatial network diversity scores were strongly correlated with IMD rank, we found a weaker positive correlation present using number of contacts and a negative correlation for communication volume."

"For example, whereas inhabitants of Stoke-on-Trent, one of the least prosperous regions in the UK, averaged a higher monthly call volume than the national average, they have one of the lowest diversity scores in the country. Similarly prosperous Stratford-upon-Avon has inhabitants with extremely diverse networks, despite no more communication than the national average. "




Predicting Spending Behavior Using Socio-mobile Features:
%http://ieeexplore.ieee.org/xpl/login.jsp?tp=&arnumber=6693330&url=http%3A%2F%2Fieeexplore.ieee.org%2Fxpls%2Fabs_all.jsp%3Farnumber%3D6693330
free version:
%https://scholar.google.com/citations?view_op=view_citation&hl=en&user=Ef1hJ8IAAAAJ&sortby=pubdate&citation_for_view=Ef1hJ8IAAAAJ:NaGl4SEjCO4C

Social behavior can be used to predict spending behavior in couples in regards to their prepensity to diversify the businesses they explore, become loyal customers and overspend. The results show that mobile phone social interaction patters can be more predictive than personality based features when predicting spending behavior. 

"We find that social behavior measured via face-to-face interaction, call, and SMS logs, can be used to predict the spending behavior for couples in terms of their propensity to explore diverse businesses, become loyal customers, and overspend"

"results show that mobile phone based social interaction patterns can provide more predictive power on spending behavior than personality based features. Interestingly, we find that more social couples also tend to overspend."




Money Walks: Implicit Mobility Behavior and Financial Well-Being:
%http://journals.plos.org/plosone/article?id=10.1371/journal.pone.0136628

Spatiotemporal traits such as exploration, engagement and elasticity can be used to predict future finanical difficulties. 

"Hence, in this work we study a large-scale cross-sectional dataset of human spending across space and time, and connect it to the biological phenomena of “foraging,” a basic pattern of animal movement to gather foods and resources."

"we analyzed a corpus of hundreds of thousands of human economic transactions and found that financial outcomes for individuals are intricately linked with their spatiotemporal traits like exploration, engagement, and elasticity. Such features yield models that are 30\% to 49\% better at predicting future financial difficulties than the comparable demographic models."

"As shown in Fig 2, individuals with lower levels of education (High School, Middle School, or Primary School) were found to be more likely to be late for their payments and get into financial trouble. Users with higher age were marginally less likely to overspend, miss payments, or get into financial trouble. Last, male customers and married customers were less likely to miss their payments."

"The figure also shows that multiple mobility behavior features were statistically correlated with outcome variables, even after controlling for the effect of abovementioned demographic variables of age, gender, marital status, education, and work type."

"the behavioral features were found to be more significantly associated (in terms of p-values) and contain higher predictive power (in terms of odds ratios being further away from 1.0 in either direction) as compared to the demographic features."

"The evidence so far indicating that each of the spatio-temporal behavioral descriptors has significant association with different financial outcomes motivates their combination to predict the financial outcome"




Predicting personality using novel mobile phone-based metrics
%http://dl.acm.org/citation.cfm?id=2456492
free version:
%https://www.google.com/url?sa=t&rct=j&q=&esrc=s&source=web&cd=1&ved=0CCEQFjAAahUKEwivqbnj0L3IAhWIcT4KHZtqCOg&url=http%3A%2F%2Frealitycommons.media.mit.edu%2Fdownload.php%3Ffile%3DdeMontjoye2013predicting-citation.pdf&usg=AFQjCNGDDMiemMv9vPQGBd_NNP30sEr6MQ&sig2=CQk3pVIzJ3EmXlV0lD8oKA

"Using a set of novel psychology-informed indicators that can be computed from data available to all carriers, we were able to predict users’ personality with a mean accuracy across traits of 42% better than random, reaching up to 61% accuracy on a three-class problem."

"The goal of the present research is to show that users’ personalities can be reliably inferred from basic information accessible from all mobile phones and to all service providers."

"The model predicted whether phone users were low, average, or high in neuroticism, extraversion, conscientiousness, agreeableness, and openness with an accuracy of 54%, 61%, 51%, 51%, and 49%, respectively."


