\section{Introduction}

Speculative bubbles , since at least the dutch tulip mania (\cite{garber2001famous} pp 127-31 for references) periodically take over markets. %note quite what he says; fundamentals driven then morals story.
The public notoriety of Bitcoin and the massive price increases that it has seen relative to its starting prices a few eyars ago have lead to an explosion of attempts to create \''the next bitcoin\'' often refered to as cryptocurrencies.
While speculation during bubbles as a social process that has been theoretically studied \cite{abolafia1988enacting, earl2007decision, bakker2010social, harras2011grow}, data on a real social network underpinning it has not been used.
The attempts to introduce these new \''coins\'' largely take place in a forum called Bitcoin talk.
We present a novel dataset constructed based on this that allows us to identify the introducers of each coin and build measures of their position in the network based on whcih users have engaged with whom in the forum before the coin is announced or traded (TODO clarify on when which).
By considering the comunity structure that exists in the forum before a coin is introduced we are able to sidestep problems of reverse causation that would plague a analisys that relied on in time variaiton between prices and network structure. 
We also collect the set of prices and traded volumes accross the cryptocurrencies that are itnroduced in the disucssions on the forums, and construct measures of both the intensity (aka severity, for each dollar invested at the peak what could be recovered on average) and the magnitude (how many dollars or bitcoins where nominally traded in the asset).
This allows us to evaluate the predictive power of the features of the node in our constructed network that corresponds to the user who first introduces the coin. 
While the magnitude of the assets traded is small relative to most financial and commodity markets, it is much larger than even the most lavishly funded experimenter could hope for.
Furthermore, the rich market structure that surrounds (some 45 exchanges appear on the dataset, raning in credibility from VC backed and registered int he US, to anonimous and mysteriously run) provides a rich source of institutional variaiton with extremely open data, a striking constrast to most financial or commodities market trade level data. 
Our contribution aims to beging in the computational social sciences a field that would have a place in the sociology of markets analogous to that of computational imaging lesion studies do in neuropsychology \footnote{ \cite[cosma2008]} .

The evidence based uncovered by “— traces of their communicative interactions as they work out their thoughts about matters of common concern” 


Bubble forecasting in economics and finance has used features of the time series (TODO cites)

Agents structural strength within the disocurse surrounding cryotographic currencies is particularly important, as these are allmost entirely perofmrative; the initial marketplace adoption of bitcoin is 
crash of  %87 http://www.iijournals.com/doi/abs/10.3905/jpm.1989.409242?journalCode=jpm 
“rapid rise of options” witht he rise of the thoery to price those options, in having currencies with zero entrie costs
MacKenzie, Donald. "An Engine, Not a Camera."
While states can create the demand required for a currency system to run by compelling tax payment in it (for a recent example), non state sponsored currencies must find some other wyas of creating demand.
The initial market for which bitcoin has been used (prices denominated in it, transactions only in it) where drug sales. citation.
New currencies have thus been floated with every single drug name posible. Many chains can claim to the same claim, so exchanges (since speculation is the only possible use of allmost all of them) become de-facto stablishers of who has a minimaly viable claim. 










