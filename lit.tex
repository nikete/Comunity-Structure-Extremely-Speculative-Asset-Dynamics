\section{Literature}

This work is at the intersection of three literatures: in economics and finance on the study of speculative bubbles, in network science on the prediction of outcomes based on features of an individual in a network, and in computer science, largely centered on the security community, studying cryptocurrencies\cite{bonneau2014mixcoin, ron2014did}.


\subsection{Bubbles and Herds}

Perhaps the most striking line of research on bubbles in economics with respect to cryptocurrencies is the study of markets where the asset is worthless and this is common knowledge. 
Recently \cite{moinas2013bubble} studies both theoretically and experimentally in the laboratory such a bubble. 
The driving force is that some traders ``do not know where they stand in the market sequence, the game allows for
a bubble at the Nash equilibrium when there is no cap on the maximum price.''.
In the context of cryptocurrencies the lack of knowledge around the sequence position maps to uncertainty about ones place in the technology adoption knowledge and adoption curve, while the difficulty in upper bounding the potential market value of cryptocurrencies provides the lack of cap on the maximum price. 


A large literature in finance empirically examines herding by financial analysts, for a recent example\cite{jegadeesh2009analysts} tests the hypothesis of herding in analysts forecasts. 

It also looks at the properties of analysts who disagree with their peers and of their forecasts

 \cite{clement2005financial} 
weather analysts are likely to disagree with their peers for a prominent recent example classifies analysts' earnings forecasts as herding or bold and finds that (1) boldness likelihood increases with the analyst's prior accuracy, brokerage size, and experience and declines with the number of industries the analyst follows, consistent with theory linking boldness with career concerns and ability; (2) bold forecasts are more accurate than herding forecasts; and (3) herding forecast revisions are more strongly associated with analysts' earnings forecast errors (actual earnings—forecast) than are bold forecast revisions. Thus, bold forecasts incorporate analysts' private information more completely and provide more relevant information to investors than herding forecasts."


\subsection{Bitcoin and Cryptocurrencies}


 heuristic clustering to group Bitcoin wallets based on evidence of shared authority, and then using re-identification attacks (i.e., empirical purchasing of goods and services) to classify the operators of those clusters. From this analysis, we characterize longitudinal changes in the Bitcoin market, the stresses these changes are placing on the system, and the challenges for those seeking to use Bitcoin for criminal or fraudulent purposes at scale." 
\cite{meiklejohn2013fistful}

fistful of bitcoins
Bitcoin is a purely online virtual currency, unbacked by either physical commodities or sovereign obligation; instead, it relies on a combination of cryptographic protection and a peer-to-peer protocol for witnessing settlements. Consequently, Bitcoin has the unintuitive property that while the ownership of money is implicitly anonymous, its flow is globally visible. In this paper we explore this unique characteristic further, using heuristic clustering to group Bitcoin wallets based on evidence of shared authority, and then using re-identification attacks (i.e., empirical purchasing of goods and services) to classify the operators of those clusters. From this analysis, we characterize longitudinal changes in the Bitcoin market, the stresses these changes are placing on the system, and the challenges for those seeking to use Bitcoin for criminal or fraudulent purposes at scale.

\cite{soska2015measuring}
Measuring the longitudinal evolution of the online anonymous marketplace ecosystem

\cite{ron2014did}
How Did Dread Pirate Roberts Acquire and Protect His Bitcoin Wealth
