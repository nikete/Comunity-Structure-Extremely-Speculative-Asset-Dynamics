This work is at the intersection of three literatures: in economics and finance on the study of speculative bubbles, in network science on the prediction of outcomes based on structural properties of an individual node in a network, and in computer science, largely centered on the security community, studying cryptocurrencies.
We provide pointers to the literature here and throughout the text. Nevertheless we refer readers to the textbook \cite{KleinbergNetworks} for an exhaustive explanation of our network features and their predictive power.

Perhaps the most striking line of research on bubbles in economics with respect to cryptocurrencies is the study of markets where the asset is worthless and this is common knowledge. 
Recently \cite{moinas2013bubble} studies both theoretically and experimentally in the laboratory such a bubble. 
The driving force is that some traders ``do not know where they stand in the market sequence, the game allows for
a bubble at the Nash equilibrium when there is no cap on the maximum price''.
In the context of cryptocurrencies the lack of knowledge around the sequence position maps to uncertainty about one's place in the technology adoption knowledge and adoption curve, while the difficulty in upper bounding the potential market value of cryptocurrencies provides the lack of cap on the maximum price. 
A large literature in finance empirically examines herding by financial analysts. For example, a recent study \cite{jegadeesh2009analysts} tests the hypothesis of herding in analysts forecasts. 
Authors of \cite{clement2005financial} study the properties of analysts who disagree with their peers and their forecasts. They find that bold forecasters better incorporate analysts' private information and offer more relevant information to investors than herding forecasters.
While extensive this literature offers little guidance in that \textcolor{red}{it does not provide any insight on the network mechanisms that might give rise to herding behavior.}

Several academics papers have recently appeared proposing cryptocoins or potential algorithms around Bitcoin that achieve better usage properties, see for example \cite{bonneau2014mixcoin}.
A fascinating line of work \cite{meiklejohn2013fistful, soska2015measuring} seeks to understand what Bitcoins are used for, both via re-identification attacks and by scrapping the marketplaces that they are used in.

