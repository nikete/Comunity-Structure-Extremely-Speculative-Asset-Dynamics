This work is at the intersection of three literatures: in economics and finance on the study of speculative bubbles, in network science on the prediction of outcomes based on structural properties of an individual node in a network, and in computer science, largely centered on the security community, studying cryptocurrencies.
We provide pointers to the literature here and throughout the text. Nevertheless we refer readers to the textbook \cite{KleinbergNetworks} for an exhaustive explanation of our network features and their predictive power.

Perhaps the most striking line of research on bubbles in economics with respect to cryptocurrencies is the study of markets where the asset is worthless and this is common knowledge. 
Recently \cite{moinas2013bubble} studies both theoretically and experimentally in the laboratory such a bubble. 
The driving force is that some traders ``do not know where they stand in the market sequence, the game allows for
a bubble at the Nash equilibrium when there is no cap on the maximum price''.
In the context of cryptocurrencies the lack of knowledge around the sequence position maps to uncertainty about one's place in the technology adoption knowledge and adoption curve, while the difficulty in upper bounding the potential market value of cryptocurrencies provides the lack of cap on the maximum price. 
A large literature in finance empirically examines herding by financial analysts. For example, a recent study \cite{jegadeesh2009analysts} tests the hypothesis of herding in analysts forecasts. 
Authors of \cite{clement2005financial} study the properties of analysts who disagree with their peers and their forecasts. They find that bold forecasters better incorporate analysts' private information and offer more relevant information to investors than herding forecasters.
While extensive this literature offers little guidance in that it does not provide any insight on the network mechanisms that might give rise to pricing effect or herding behavior.

Perhaps, the most exhaustive work on the effect of social networks on economic outcomes is \cite{Granovetter-outcomes}. In \cite{Granovetter-outcomes} notes that institutionalized acceptance of financial derivatives as an economic innovation in Chicago Board of Trades was only possible since their proponents used their social status and intensive interactions with CBT insiders to promote collective acceptance of derivatives. This insight has striking similarities with our work since one could view altcoins as innovations whose success heavily depends on the social status of their proponents. \cite{Granovetter-outcomes} also mentions economic scenarios where majority of actors trade preferentially and social relations, in particular trust, between actors are used to overcome adverse selection thus affecting prices.
In a contrasting empirical setting setting to ours, both due to the value the adoption brings to the adoptee and the type of technology, \cite{maertens2013measuring} review the literature of social networks effects on agricultural technology adoption.

Several academics papers have recently appeared studying Bitcoin price volatility and its relationship with market sentiment. The most notable work by \cite{Garcia-bitcoin} uses various aggregate social insights such word-of-mouth volume and valence on social media to design an optimal Bitcoin trading strategy . In contrast to \cite{Garcia-bitcoin}, we focus on micro-level information in the network of active cryptocurrency users and use the pricing data on a large collection of altcoins rather than a single currency (Bitcoin). Other recent papers have proposed cryptocoins or potential algorithms around Bitcoin that achieve better usage properties, see for example \cite{bonneau2014mixcoin}.
A fascinating line of work \cite{meiklejohn2013fistful,soska2015measuring} seeks to understand what Bitcoins are used for, both via re-identification attacks and by scrapping the marketplaces that they are used in.

