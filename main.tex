% THIS IS SIGPROC-SP.TEX - VERSION 3.1
% WORKS WITH V3.2SP OF ACM_PROC_ARTICLE-SP.CLS
% APRIL 2009
%
% It is an example file showing how to use the 'acm_proc_article-sp.cls' V3.2SP
% LaTeX2e document class file for Conference Proceedings submissions.
% ----------------------------------------------------------------------------------------------------------------
% This .tex file (and associated .cls V3.2SP) *DOES NOT* produce:
%       1) The Permission Statement
%       2) The Conference (location) Info information
%       3) The Copyright Line with ACM data
%       4) Page numbering
% ---------------------------------------------------------------------------------------------------------------
% It is an example which *does* use the .bib file (from which the .bbl file
% is produced).
% REMEMBER HOWEVER: After having produced the .bbl file,
% and prior to final submission,
% you need to 'insert'  your .bbl file into your source .tex file so as to provide
% ONE 'self-contained' source file.
%
% Questions regarding SIGS should be sent to
% Adrienne Griscti ---> griscti@acm.org
%
% Questions/suggestions regarding the guidelines, .tex and .cls files, etc. to
% Gerald Murray ---> murray@hq.acm.org
%
% For tracking purposes - this is V3.1SP - APRIL 2009

\documentclass{acm_proc_article-sp}

\begin{document}

\title{Comunity Structure & Extremely Speculative Asset Dynamics}
%
% You need the command \numberofauthors to handle the 'placement
% and alignment' of the authors beneath the title.
%
% For aesthetic reasons, we recommend 'three authors at a time'
% i.e. three 'name/affiliation blocks' be placed beneath the title.
%
% NOTE: You are NOT restricted in how many 'rows' of
% "name/affiliations" may appear. We just ask that you restrict
% the number of 'columns' to three.
%
% Because of the available 'opening page real-estate'
% we ask you to refrain from putting more than six authors
% (two rows with three columns) beneath the article title.
% More than six makes the first-page appear very cluttered indeed.
%
% Use the \alignauthor commands to handle the names
% and affiliations for an 'aesthetic maximum' of six authors.
% Add names, affiliations, addresses for
% the seventh etc. author(s) as the argument for the
% \additionalauthors command.
% These 'additional authors' will be output/set for you
% without further effort on your part as the last section in
% the body of your article BEFORE References or any Appendices.

\numberofauthors{8} %  in this sample file, there are a *total*
% of EIGHT authors. SIX appear on the 'first-page' (for formatting
% reasons) and the remaining two appear in the \additionalauthors section.
%
\author{
% You can go ahead and credit any number of authors here,
% e.g. one 'row of three' or two rows (consisting of one row of three
% and a second row of one, two or three).
%
% The command \alignauthor (no curly braces needed) should
% precede each author name, affiliation/snail-mail address and
% e-mail address. Additionally, tag each line of
% affiliation/address with \affaddr, and tag the
% e-mail address with \email.
%
% 1st. author
\alignauthor
Nicolas Della Penna\titlenote{Dr.~Trovato insisted his name be first.}\\
       \affaddr{Institute for Clarity in Documentation}\\
       \affaddr{1932 Wallamaloo Lane}\\
       \affaddr{Wallamaloo, New Zealand}\\
       \email{trovato@corporation.com}
% 2nd. author
\alignauthor
Eaman \titlenote{is first co-author.}\\
       \affaddr{Institute for Clarity in Documentation}\\
       \affaddr{P.O. Box 1212}\\
       \affaddr{Dublin, Ohio 43017-6221}\\
       \email{webmaster@marysville-ohio.com}
% 3rd. author
\alignauthor Peter Kraft \titlenote{This author is the
one who did all the really hard work.}\\
       \affaddr{The Th{\o}rv{\"a}ld Group}\\
       \affaddr{1 Th{\o}rv{\"a}ld Circle}\\
       \affaddr{Hekla, Iceland}\\
       \email{larst@affiliation.org}
\and  % use '\and' if you need 'another row' of author names
% 4th. author
\alignauthor Lawrence P. Leipuner\\
       \affaddr{Brookhaven Laboratories}\\
       \affaddr{Brookhaven National Lab}\\
       \affaddr{P.O. Box 5000}\\
       \email{lleipuner@researchlabs.org}
% 5th. author
\alignauthor Sean Fogarty\\
       \affaddr{NASA Ames Research Center}\\
       \affaddr{Moffett Field}\\
       \affaddr{California 94035}\\
       \email{fogartys@amesres.org}
% 6th. author
\alignauthor Charles Palmer\\
       \affaddr{Palmer Research Laboratories}\\
       \affaddr{8600 Datapoint Drive}\\
       \affaddr{San Antonio, Texas 78229}\\
       \email{cpalmer@prl.com}
}
% There's nothing stopping you putting the seventh, eighth, etc.
% author on the opening page (as the 'third row') but we ask,
% for aesthetic reasons that you place these 'additional authors'
% in the \additional authors block, viz.
\date{today}
% Just remember to make sure that the TOTAL number of authors
% is the number that will appear on the first page PLUS the
% number that will appear in the \additionalauthors section.

\maketitle
\begin{abstract}


\end{abstract}

% A category with the (minimum) three required fields
%\category{H.4}{Information Systems Applications}{Miscellaneous}
%A category including the fourth, optional field follows...
%\category{D.2.8}{Software Engineering}{Metrics}[complexity measures, performance measures]

\terms{Theory}

\keywords{ACM proceedings, \LaTeX, text tagging} % NOT required for Proceedings


\section{Introduction}

Speculative bubbles , since at least the dutch tulip mania (Shiller 2005) periodically take over markets.
While speculation during bubbles as a social process that has been theoretically studied \cite{abolafia1988enacting, earl2007decision, bakker2010social} data on the social network has not been used in such studies. 
The public notoriety of Bitcoin and the massive price increases that it has seen relative to its starting prices a few eyars ago have lead to an explosion of attempts to create \''the next bitcoin\''.
The attempts to introduce these new \''coins\'' largely take place in a forum called Bitcoin talk, we present a novel dataset constructed on this data that allows us to identify the introducers of each coin 

https://books.google.com.au/books?id=0TsEM40mepQC&pg=PA144&lpg=PA144&dq=network+structure+and+bubbles&source=bl&ots=3F0pQ7LRPY&sig=JDO3QCEVqJX7KRN0h8Qltm7KgVQ&hl=en&sa=X&ved=0CCwQ6AEwAmoVChMIt_Tz1LHByAIVhqGUCh0x6AoA#v=onepage&q=network%20structure%20and%20bubbles&f=false

The explosion of cryptocurrencies that follows, in their online comunities and 
and the dynamics of the markets around it provide a chance to observe highly detailed data on a cross ection of comparable asset bubbles.
We collect and make public a dataset of the bitcointalk discussion forum (posts, authors) and use to construct a graph of whcih users have engaged with whom in the forum. 
We also collect a  set of prices and traded volumes accross the cryptocurrencies that are itnroduced in the disucssions on the forums.
This allows us to evaluate the predictive power 

A simple 

Provide time separation argument

Provide 


We study the power of structural features of the social network around cryptocurrencies to understand the severity of bubbles that occur in them. 
Our goal is to asses the predictive capacity of structural featres of derived from the social networks that completely preceed tradingin the asset.
Bubble forecasting in economics and finance has used features of the time series
 All our measures are constructed using observations to contruct a network before the relevant cryptocurrency is ever traded. 



“ it would have a place in the sociology of markets analogous to that of lesion studies in neuropsychology.” a window into the social life of the mind https://www.aaai.org/Papers/Symposia/Spring/2008/SS-08-06/SS08-06-017.pdf

The evidence based uncovered by “— traces of their communicative interactions as they work out their thoughts about matters of common concern” 

crash of 87 http://www.iijournals.com/doi/abs/10.3905/jpm.1989.409242?journalCode=jpm “rapid rise of options” witht he rise of the thoery to price those options, in having currencies with zero entrie costs
MacKenzie, Donald. "An Engine, Not a Camera."
While states can create the demand required for a currency system to run by compelling tax payment in it (for a recent example), non state sponsored currencies must find some other wyas of creating demand.
The initial market for which bitcoin has been used (prices denominated in it, transactions only in it) where drug sales. citation.
New currencies have thus been floated with every single drug name posible. Many chains can claim to the same claim, so exchanges (since speculation is the only possible use of allmost all of them) become de-facto stablishers of who has a minimaly viable claim. 


"We develop a series of cross-sectional regression specifications to forecast skewness in the daily returns of individual stocks. Negative skewness is most pronounced in stocks that have experienced (1) an increase in trading volume relative to trend over the prior six months, consistent with the model of Hong and Stein (NBER Working Paper, 1999), and (2) positive returns over the prior 36 months, which fits with a number of theories, most notably Blanchard and Watson's (Crises in Economic and Financial Structure. Lexington Books, Lexington, MA, 1982, pp. 295–315) rendition of stock-price bubbles. Analogous results also obtain when we attempt to forecast the skewness of the aggregate stock market, though our statistical power in this case is limited."






Data for the historical valuations of the coins was scrapped from coinmarketcap.com The bubble measure was constructed as the log of the ratio in the loss of value from the maximum notional valuation of the available currency supply to to the present value. 

Our network measurement data consisted of discussions in Bitcoin and Altcoin online forums on bitcointalk.org from January 2010 until July 2014. 
Bitcoin forum is mainly concerned with Bitcoin and related technological issues whereas Altcoin discussions are concerned about alternative cryptocurrencies. Each forum consists of hundreds of threads and each thread contains many posts.  Given the forum discussion data at the level of individual posts, we constructed the undirected graph of user discussions. In this network, nodes are the forum users and are connected by an edge if the corresponding users have ever interacted in any thread. The edge weights between two users are determined based on their interaction frequency. Furthermore, the weights are adjusted by size of the thread (i.e. an interaction in small threads counts more than an interaction in a large thread) and a decay factor (i.e. a recent interaction counts more than an old interaction). We constructed the network over time by replaying the posts and updating the discussion graph accordingly.  Whenever a new digital currency was introduced in the forum for the first time, a snapshot of the graph was taken and used for extracting various network measures corresponding to the coin. It is important to analyze the discussion graph before the coin is mentioned for the first time to avoid any possible confounding factors. The network measures we used were as followed: number of edges, number of nodes, density, diameter, clustering coefficient, average path length and average degree. We also computed the following network measures for the user who introduced the crypto coin for the first time: clustering coefficient, closeness centrality, betweenness centrality and laplacian centrality. As future work, we will construct the directed “Satoshi” network at time of the coin introduction where an edge exists from user a to user b, if user a has ever replied to user b. Relevant features of this network such as “Satoshi” distance of the coin introducer and her eigenvector centrality can be used in analysis of bubbles.
A OLS fit of bubble severity to the network structure measures fails to provide statistically significant explanatory power. Inspection of the model fit reveals that the model does however successfully estimates high bubble values for the most severe bubbles, and there is never a severe bubble when it predicts low severity.


Tâtonnement stability of infinite horizon models with saddle-point instability
WP Heller - Econometrica: Journal of the Econometric Society, 1975 - JSTOR
... horizon case. 5 As Samuelson [17, p. 981] notes, ". . . history tells us that all tulip
manias have ended in finite time.... Every bubble is someday pricked." Knowing this,
people start to sell when prices reach ridiculous heights. But at ...






\section{Literature}

This work is at the intersection of three literatures: economics on the study of speculative bubbles, network science on the prediction of outcomes based on features of an individual in a network, and a nacent field in the computer science security comunity that studies cryptocurrencies.


\subsection{Bubbles}

Perhaps the most striking line of research on bubbles in economics with respect to the cryptocurrency market is the study of those where the asset is worthless and this is common knowledge. 

In \cite{} " a bubble game that involves sequential trading of an
asset commonly known to be valueless. Because some traders do not
know where they stand in the market sequence, the game allows for
a bubble at the Nash equilibrium when there is no cap on the maximum
price. We run experiments both with and without a price cap.
Structural estimation of behavioral game theory models suggests that
quantal responses, uncertainty regarding other traders’ rationality,
and analogy-based expectations are important drivers of speculation. "


Decision-rule cascades and the dynamics of speculative bubbles
We combine Minsky’s Wnancial fragility analysis, behavioural analysis of decision rules and the
evolutionary economics of rule trajectories to provide an empirically grounded and computationally
tractable theory of the complex evolutionary dynamics of speculative Wnancial upswings. The behavioural
dynamics of asset bubbles can be conceptualized as the joint consequence of the adoption and
diVusion process of new investment decision rules coupled with the degradation of those rules as they
pass from a few expert investors to larger population of amateurs. We illustrate this using data covering
the recent Brisbane property market bubble (1999–2003) and show how it is consistent with the
existence of such cascading decision rules. We then explain how multi-agent simulation methods can
be used for modelling decision rule cascades.
© 2006 Elsevier B.V. All rights reserved."

\subsection{Prediction from networks}

"The Structural Virality of Online Diffusion" (Management Science)




"Here we propose a formal measure of what we label “structural virality” that interpolates between two conceptual extremes: content that gains its popularity through a single, large broadcast, and that which grows through multiple generations with any one individual directly responsible for only a fraction of the total adoption"

"We find that across all domains and all sizes of events, online diffusion is characterized by surprising structural diversity. Popular events, that is, regularly grow via both broadcast and viral mechanisms, as well as essentially all conceivable combinations of the two."

"we find that the correlation between the size of an event and its structural virality is surprisingly low, meaning that knowing how popular a piece of content is tells one little about how it spread"

"We find that while several of our empirical findings are consistent with such a model, it does not replicate the observed diversity of structural virality"











Network Diversity and Economic Development
http://www.sciencemag.org/content/328/5981/1029.full



"Hence, highly clustered, or insular, social ties are predicted to limit access to social and economic prospects from outside the social group, whereas heterogeneous social ties may generate these opportunities from a range of diverse contacts (1, 2)."

"Although both social and spatial network diversity scores were strongly correlated with IMD rank, we found a weaker positive correlation present using number of contacts and a negative correlation for communication volume."

"For example, whereas inhabitants of Stoke-on-Trent, one of the least prosperous regions in the UK, averaged a higher monthly call volume than the national average, they have one of the lowest diversity scores in the country. Similarly prosperous Stratford-upon-Avon has inhabitants with extremely diverse networks, despite no more communication than the national average. "




Predicting Spending Behavior Using Socio-mobile Features:
%http://ieeexplore.ieee.org/xpl/login.jsp?tp=&arnumber=6693330&url=http%3A%2F%2Fieeexplore.ieee.org%2Fxpls%2Fabs_all.jsp%3Farnumber%3D6693330
free version:
%https://scholar.google.com/citations?view_op=view_citation&hl=en&user=Ef1hJ8IAAAAJ&sortby=pubdate&citation_for_view=Ef1hJ8IAAAAJ:NaGl4SEjCO4C

Social behavior can be used to predict spending behavior in couples in regards to their prepensity to diversify the businesses they explore, become loyal customers and overspend. The results show that mobile phone social interaction patters can be more predictive than personality based features when predicting spending behavior. 

"We find that social behavior measured via face-to-face interaction, call, and SMS logs, can be used to predict the spending behavior for couples in terms of their propensity to explore diverse businesses, become loyal customers, and overspend"

"results show that mobile phone based social interaction patterns can provide more predictive power on spending behavior than personality based features. Interestingly, we find that more social couples also tend to overspend."




Money Walks: Implicit Mobility Behavior and Financial Well-Being:
%http://journals.plos.org/plosone/article?id=10.1371/journal.pone.0136628

Spatiotemporal traits such as exploration, engagement and elasticity can be used to predict future finanical difficulties. 

"Hence, in this work we study a large-scale cross-sectional dataset of human spending across space and time, and connect it to the biological phenomena of “foraging,” a basic pattern of animal movement to gather foods and resources."

"we analyzed a corpus of hundreds of thousands of human economic transactions and found that financial outcomes for individuals are intricately linked with their spatiotemporal traits like exploration, engagement, and elasticity. Such features yield models that are 30\% to 49\% better at predicting future financial difficulties than the comparable demographic models."

"As shown in Fig 2, individuals with lower levels of education (High School, Middle School, or Primary School) were found to be more likely to be late for their payments and get into financial trouble. Users with higher age were marginally less likely to overspend, miss payments, or get into financial trouble. Last, male customers and married customers were less likely to miss their payments."

"The figure also shows that multiple mobility behavior features were statistically correlated with outcome variables, even after controlling for the effect of abovementioned demographic variables of age, gender, marital status, education, and work type."

"the behavioral features were found to be more significantly associated (in terms of p-values) and contain higher predictive power (in terms of odds ratios being further away from 1.0 in either direction) as compared to the demographic features."

"The evidence so far indicating that each of the spatio-temporal behavioral descriptors has significant association with different financial outcomes motivates their combination to predict the financial outcome"




Predicting personality using novel mobile phone-based metrics
%http://dl.acm.org/citation.cfm?id=2456492
free version:
%https://www.google.com/url?sa=t&rct=j&q=&esrc=s&source=web&cd=1&ved=0CCEQFjAAahUKEwivqbnj0L3IAhWIcT4KHZtqCOg&url=http%3A%2F%2Frealitycommons.media.mit.edu%2Fdownload.php%3Ffile%3DdeMontjoye2013predicting-citation.pdf&usg=AFQjCNGDDMiemMv9vPQGBd_NNP30sEr6MQ&sig2=CQk3pVIzJ3EmXlV0lD8oKA

"Using a set of novel psychology-informed indicators that can be computed from data available to all carriers, we were able to predict users’ personality with a mean accuracy across traits of 42% better than random, reaching up to 61% accuracy on a three-class problem."

"The goal of the present research is to show that users’ personalities can be reliably inferred from basic information accessible from all mobile phones and to all service providers."


"The model predicted whether phone users were low, average, or high in neuroticism, extraversion, conscientiousness, agreeableness, and openness with an accuracy of 54%, 61%, 51%, 51%, and 49%, respectively."



\section{Data Description}



\subsection{Prices and Exchanges}
TODO: NIKETE I did not remove this, I just moved it down to analysis vars. Wanted this subsection
to only talk about the data and not how you make the bubble measure.

Our main outcome measures are the severity of the inflation an asset price, and the magnitude of money transacted in it.
We operationalize the intensity of a bubble as the proportion of a 1 dollar (TODO check currency base) that would be lost buying at the maximum price and selling after that proportionally to the volume of the market till the present, we call this severity.
We define the volume as the sum of the contemporaneous dollar (todo check) volume of trade.
As a secondary outcome measure we consider the number of exchanges that list the coin.
We scraped two major price aggregators; coinmarketcap and coininfo.


\subsection{Forum Discussions}
TODO: EAMAN WILL WRITE THIS

In order to study the effect of communication network around cryptocoins on
price variations, we collected all the posts from the most famous cryptocurrency
online community, bitcointalk.  Our data consisted of all the posts that were
made between January 2010 and July 2015 on the most acitve crypto-related forums:
\begin{enumerate}
  \item{Bitcoin Discussion} This is the oldest forum on the website which mainly focuses
    on issues only related to Bitcoin. Interestingly, Satoshi Nakamoto, the alleged
    creator of Bitcoin made the first post on this forum in January 2010 and
    was active until January 2011. The presence of Satoshi in the data set enables us
    to study the position of various actors in the online community relative to Satoshi
    and its relation with the success or failure of cryptocoins they advocate or reject.

  \item{Altcoin Discussion} This is the most active forum in the community
    with more than 730000 posts as of July 2015, and dating back to June 2011.
    The discussions in this forum mainly evolve around alternative currencies
    other than Bitcoin. Users often discuss the merits or flaws of various
    altcoins or simply exchange technical information.
  
  \item{Announcement (Altcoin)} Community announcements such as development of 
    exchange client or addition of new features are made here. This is an important forum
    in our study as the creation of new altcoins are announced here. Whenever a new
    altcoin is announced to the community, the announcement is tagged with string ANN.
    This enables us to detect announcement of new coins into the market and identify
    the users who introduced them for the first time.

  \item{Mining (Altcoin)} Technical issues pertaining to mining (i.e. validating transactions)
    altcoins are discussed here.
  \item{Marketplace (Altcoin)} This forum contains the discussions on a wide-range of 
    market-related issues, such as price or volume trends, possible pump and dump schemes
    and exchange tips.

\end{enumerate}
TODO: CHECK AVERAGE NUMBER OF POSTS PER THREAD, and other statistics
Each forum consists of many subjects or threads initiated by different users. Each thread
contains several posts or replies, with an average of 10 posts per thread. Each post has several fields
which contain valuable information in our context.
\begin{enumerate}
  \item{Subject}: Usually, the initiator of the thread chooses subject and all the
    following posts inherit the same subject.
  \item{Content}: The acutal text of the message.
  \item{Author}
  \item{Date}
  \item{Position in the thread}: The later posts in the thread might not be as important
    as earlier posts and could be about issues other than the original topic of the thread.
\end{enumerate}

The community had only 10000 unique users until early 2013, however the
community size grew considerably faster after 2013 and reached 70000 by early
2015. Neverthelss, there are only 10000 active users within any 30 day period on average.


\section{Forum Interaction Network}
TODO: EAMAN WILL WRITE THIS
TODO: TALK ABOUT UNDIRECTED NETWORK WITHOUT RESULTS?

The reply structure within each thread constitutes the basis of our discussion network, discussed later.

Given the forum discussion data at the level of individual posts, we
constructed the undirected graph of user discussions. In this network, nodes
are the forum users and are connected by an edge if the corresponding users
have ever interacted in any thread. The edge weights between two users are
determined based on their interaction frequency. Furthermore, the weights are
adjusted by size of the thread (i.e. an interaction in small threads counts
more than an interaction in a large thread) and a decay factor (i.e. a recent
interaction counts more than an old interaction).

We construct the network over time by replaying the posts and updating the
discussion graph accordingly.  Whenever a new digital currency was introduced
in the forum for the first time, a snapshot of the graph was taken and used for
extracting various network measures corresponding to the coin. It is important
to analyze the discussion graph before the coin is mentioned for the first time
to avoid any possible confounding factors. The network measures we used were as
followed: number of edges, number of nodes, density, diameter, clustering
coefficient, average path length and average degree. We also computed the
following network measures for the user who introduced the crypto coin for the
first time: clustering coefficient, closeness centrality, betweenness
centrality and laplacian centrality.



\section{Analysis Variables}
\subsection{Prices and Exchanges}
Our main outcome measures are the severity of the inflation an asset price, and the magnitude of money transacted in it.
We operationalize the intensity of a bubble as the proportion of a 1 dollar (TODO check currency base) that would be lost buying at the maximum price and selling after that proportionally to the volume of the market till the present, we call this severity.
We define the volume as the sum of the contemporaneous dollar (todo check) volume of trade.
As a secondary outcome measure we consider the number of exchanges that list the coin.


\subsection{Network Structure}
TODO: EAMAN WILL WRITE THIS
TODO: ENUMERATE ALL MEASURES

\input{methodology.tex}
We fitted $7$ models with our methodology for each of our two outcome variables constructed based on the first user who used either the coin name or symbol in an ANN tagged thread.
Tables 1 and 2 report the coefficient estimates for standardized variables (both outcomes and measures have been linearly transformed to have mean 0 and standard deviation 1).
As a robustness check in the appendix tables 3 and 4 show the same results when applied to a 

Our User model considers how long a user has been active, and how many users have replied to them and he has replied to. 
Our nontrivial model consists of a single binary variable encoding wether the source code of the coin is a not a trivial change to existing coins. 
Our Satoshi model considers the distance and pagerank relative to satoshi.
None  of these three models achieve out of sample error rates beyond simply predicting $0$ constantly for either of our outcome measures of interest.


Our Network model encodes directed edges when a user interacts with another, and proceeds to estimate closeness centrality and clustering measures based on this. 
Our Weighted model uses the intensity of interactions to create a weighted network and construct a similar set of measures to the Network model.
Our interaction model includes interaction terms between non-trivialness and the weighted network measures.
Our All model contains all previous mentioned variables, there is extreme collinearity in this model and thus it's estimated coefficients and standard errors should be interpreted with extreme caution. 


\begin{figure}[h]
\centering
\begin{subfigure}
{\includegraphics[width=\columnwidth]{centrality_volume_severity.pdf}}
\end{subfigure}
\begin{subfigure}
{\includegraphics[width=\columnwidth]{centrality_volume_severity_nontrivial}}
\end{subfigure}
\caption{Interaction of Severity and Magnitude metrics with introducer closeness centrality (top), segmented by nontrivial classification (below). \textcolor{red}{colorbar label needs to change to introducer closeness centrality, switch bottm plot to pdf version. But for some reason left subplot contours don't show up.}}
\label{centrality_volume_severity}
\end{figure}

The main driver of our explanatory power is the centrality of a user in the directed network derived from the forum. 
This effect appears to be mediated by whether a coin involves a nontrivial technological change, the direction of the interaction reversing depending on whether it relates to magnitude or severity.
While the linear model is not able to fully capture this interaction, plotting the values of severity and magnitude of bubbles and the centrality reveals a interaction; with the polarity of the gradients flipping in the severity case.

%JULIAN: Above sentence is tough to parse, but kind of an important one.
Both the severity and the magnitude of bubbles increases with the centrality of the user who introduces the coins in the forums.  
Interestingly this effect is concentrated in different ways depending on whether the coin software is more than a trivial modification: trivial coins have more severe bubbles the more central their introducers are, while volume is greater the more central the introducer of a nontrivial coin is.






\begin{figure}[h]
\centering
\begin{subfigure}
{\includegraphics[width=\columnwidth]{cluster_closeness_volume.pdf}}
\vspace{-2\baselineskip}
\end{subfigure}
\begin{subfigure}
{\includegraphics[width=\columnwidth]{cluster_closeness_severity.pdf}}
\end{subfigure}
\caption{Interaction of introducer closeness centrality and clustering coefficient with bubble Magnitude (top) and Severity (bottom). \textcolor{red}{colorbar label needs to change to Magnitude and Severity.}}
\label{cluster_closeness_volume}
\end{figure}


\section{Conclusion}


The total variance accounted for is small, so you need a discussion like "results suggest that bubble dynamics may be strongly influenced by a core set of participants, but that traditional network measures on the aggregate discussion graph do not provide a tight characterization of this core group.  We are looking at coarse (positive/negative, detailed/cursory) semantic analysis of the discussion, and evidence of prior cooperation between pairs of participants in other altcoin markets, in order to attempt this more accurate characterization of core participants and their actions.



\section{Data}



\section{}



\section{Acknowledgments}

%
% The following two commands are all you need in the
% initial runs of your .tex file to
% produce the bibliography for the citations in your paper.
\bibliographystyle{abbrv}
\bibliography{sigproc}  % sigproc.bib is the name of the Bibliography in this case
% You must have a proper ".bib" file
%  and remember to run:
% latex bibtex latex latex
% to resolve all references
%
% ACM needs 'a single self-contained file'!
%
%APPENDICES are optional
%\balancecolumns
\appendix
%Appendix A
\section{Headings in Appendices}
The rules about hierarchical headings discussed
\balancecolumns
% That's all folks!
\end{document}
