\documentclass{acm_proc_article_sp}% http://www.acm.org/sigs/publications/acm_proc_article-sp.cls/view
%\usepackage{graphicx}
\graphicspath{ {images/} }
%\usepackage[section]{placeins}
%\usepackage{pdflscape}
\usepackage{enumitem}
\usepackage{color}
\usepackage{subfigure}
\usepackage{hyperref}

\begin{document}
\title{Forever Blowing Bitcoins: Social Structure and Speculative Bubbles in Cryptocurrencies}

\numberofauthors{6} %  in this sample file, there are a *total*
% of EIGHT authors. SIX appear on the 'first-page' (for formatting
% reasons) and the remaining two appear in the \additionalauthors section.
%
\author{
% You can go ahead and credit any number of authors here,
% e.g. one 'row of three' or two rows (consisting of one row of three
% and a second row of one, two or three).
%
% The command \alignauthor (no curly braces needed) should
% precede each author name, affiliation/snail-mail address and
% e-mail address. Additionally, tag each line of
% affiliation/address with \affaddr, and tag the
% e-mail address with \email.
%
% 1st. author
\alignauthor
Nicolas Della Penna \thanks{indicates equal contributions.}\\
%Nicolas Della Penna \titlenote{is a first co-author of this paper.}\\% He conceived the study, methodology, and wrote the initial manuscript and the notebook with the data analysis.}\\
       \affaddr{ANU}\\
       \email{n@nikete.com}
% 2nd. author
\alignauthor
Eaman Jahani \footnotemark[1] \\
%Eaman Jahani \titlenote{is a first co-author of this paper.}\\% He wrote the forum scrapers, constructed networks and their measures, and wrote up the relevant sections in network data and variables.}
       \affaddr{MIT} \\
        \email{eaman@mit.edu}
% 3rd. author
\alignauthor
Peter Krafft \\%\titlenote{helped with methodology and literature review} \\
       \affaddr{MIT} \\
       \email{pkrafft@mit.edu}
\and  % use '\and' if you need 'another row' of author names
\alignauthor
Octavio Bunge \\%\titlenote{ wrote scrapers for coin prices, and the non-trivialness measure.}\\
       \affaddr{Universidad de Belgrano} \\
        \email{octavio.bunge@\\comunidad.ub.edu.ar}
% 4th. author
\alignauthor 
Harper Reed \\%\titlenote{helped with the writing and literature review}\\
%       \affaddr{Harper Rules LLC}\\
       \email{~\\harper@nata2.org}
% 5th. author
\and  % use '\and' if you need 'another row' of author names
\alignauthor Alex (Sandy) Pentland \\ %\titlenote{helped with methodology and writing}\\
       \affaddr{MIT Media Lab}\\
       \email{pentland@mit.edu}
% 4th. author
\alignauthor  Julian McAuley \\%\titlenote{helped with methodology and writing}\\
       \affaddr{UC San Diego}\\
       \email{jmcauley@cse.ucsd.edu}
}

\maketitle

\begin{abstract}

We study the power of structural features of the social network around cryptocurrencies to understand the severity and magnitude of bubbles. 
We introduce a novel dataset that combines measures of the social network surrounding the introduction of coins in online cryptocurrency forums, the trading  behavior across marketplaces, and the presence of nontrivial changes to the source code in relation to previous coins.
Our network are constructed based on the intensity of social interactions in the forums and all the structural features of the network are measured based on the state of the social network before the relevant cryptocurrency is ever traded.
Our study reveals that structural features of the network and the nature of the source code changes achieve substantial predictive power on the severity of trading price bubbles 

\end{abstract}



\section{Introduction}
\section{Introduction}

Speculative bubbles , since at least the dutch tulip mania (Shiller 2005) periodically take over markets.
While speculation during bubbles as a social process that has been theoretically studied \cite{abolafia1988enacting, earl2007decision, bakker2010social} data on the social network has not been used in such studies. 
The public notoriety of Bitcoin and the massive price increases that it has seen relative to its starting prices a few eyars ago have lead to an explosion of attempts to create \''the next bitcoin\''.
The attempts to introduce these new \''coins\'' largely take place in a forum called Bitcoin talk, we present a novel dataset constructed on this data that allows us to identify the introducers of each coin 

https://books.google.com.au/books?id=0TsEM40mepQC&pg=PA144&lpg=PA144&dq=network+structure+and+bubbles&source=bl&ots=3F0pQ7LRPY&sig=JDO3QCEVqJX7KRN0h8Qltm7KgVQ&hl=en&sa=X&ved=0CCwQ6AEwAmoVChMIt_Tz1LHByAIVhqGUCh0x6AoA#v=onepage&q=network%20structure%20and%20bubbles&f=false

The explosion of cryptocurrencies that follows, in their online comunities and 
and the dynamics of the markets around it provide a chance to observe highly detailed data on a cross ection of comparable asset bubbles.
We collect and make public a dataset of the bitcointalk discussion forum (posts, authors) and use to construct a graph of whcih users have engaged with whom in the forum. 
We also collect a  set of prices and traded volumes accross the cryptocurrencies that are itnroduced in the disucssions on the forums.
This allows us to evaluate the predictive power 

A simple 

Provide time separation argument

Provide 


We study the power of structural features of the social network around cryptocurrencies to understand the severity of bubbles that occur in them. 
Our goal is to asses the predictive capacity of structural featres of derived from the social networks that completely preceed tradingin the asset.
Bubble forecasting in economics and finance has used features of the time series
 All our measures are constructed using observations to contruct a network before the relevant cryptocurrency is ever traded. 



“ it would have a place in the sociology of markets analogous to that of lesion studies in neuropsychology.” a window into the social life of the mind https://www.aaai.org/Papers/Symposia/Spring/2008/SS-08-06/SS08-06-017.pdf

The evidence based uncovered by “— traces of their communicative interactions as they work out their thoughts about matters of common concern” 

crash of 87 http://www.iijournals.com/doi/abs/10.3905/jpm.1989.409242?journalCode=jpm “rapid rise of options” witht he rise of the thoery to price those options, in having currencies with zero entrie costs
MacKenzie, Donald. "An Engine, Not a Camera."
While states can create the demand required for a currency system to run by compelling tax payment in it (for a recent example), non state sponsored currencies must find some other wyas of creating demand.
The initial market for which bitcoin has been used (prices denominated in it, transactions only in it) where drug sales. citation.
New currencies have thus been floated with every single drug name posible. Many chains can claim to the same claim, so exchanges (since speculation is the only possible use of allmost all of them) become de-facto stablishers of who has a minimaly viable claim. 


"We develop a series of cross-sectional regression specifications to forecast skewness in the daily returns of individual stocks. Negative skewness is most pronounced in stocks that have experienced (1) an increase in trading volume relative to trend over the prior six months, consistent with the model of Hong and Stein (NBER Working Paper, 1999), and (2) positive returns over the prior 36 months, which fits with a number of theories, most notably Blanchard and Watson's (Crises in Economic and Financial Structure. Lexington Books, Lexington, MA, 1982, pp. 295–315) rendition of stock-price bubbles. Analogous results also obtain when we attempt to forecast the skewness of the aggregate stock market, though our statistical power in this case is limited."






Data for the historical valuations of the coins was scrapped from coinmarketcap.com The bubble measure was constructed as the log of the ratio in the loss of value from the maximum notional valuation of the available currency supply to to the present value. 

Our network measurement data consisted of discussions in Bitcoin and Altcoin online forums on bitcointalk.org from January 2010 until July 2014. 
Bitcoin forum is mainly concerned with Bitcoin and related technological issues whereas Altcoin discussions are concerned about alternative cryptocurrencies. Each forum consists of hundreds of threads and each thread contains many posts.  Given the forum discussion data at the level of individual posts, we constructed the undirected graph of user discussions. In this network, nodes are the forum users and are connected by an edge if the corresponding users have ever interacted in any thread. The edge weights between two users are determined based on their interaction frequency. Furthermore, the weights are adjusted by size of the thread (i.e. an interaction in small threads counts more than an interaction in a large thread) and a decay factor (i.e. a recent interaction counts more than an old interaction). We constructed the network over time by replaying the posts and updating the discussion graph accordingly.  Whenever a new digital currency was introduced in the forum for the first time, a snapshot of the graph was taken and used for extracting various network measures corresponding to the coin. It is important to analyze the discussion graph before the coin is mentioned for the first time to avoid any possible confounding factors. The network measures we used were as followed: number of edges, number of nodes, density, diameter, clustering coefficient, average path length and average degree. We also computed the following network measures for the user who introduced the crypto coin for the first time: clustering coefficient, closeness centrality, betweenness centrality and laplacian centrality. As future work, we will construct the directed “Satoshi” network at time of the coin introduction where an edge exists from user a to user b, if user a has ever replied to user b. Relevant features of this network such as “Satoshi” distance of the coin introducer and her eigenvector centrality can be used in analysis of bubbles.
A OLS fit of bubble severity to the network structure measures fails to provide statistically significant explanatory power. Inspection of the model fit reveals that the model does however successfully estimates high bubble values for the most severe bubbles, and there is never a severe bubble when it predicts low severity.


Tâtonnement stability of infinite horizon models with saddle-point instability
WP Heller - Econometrica: Journal of the Econometric Society, 1975 - JSTOR
... horizon case. 5 As Samuelson [17, p. 981] notes, ". . . history tells us that all tulip
manias have ended in finite time.... Every bubble is someday pricked." Knowing this,
people start to sell when prices reach ridiculous heights. But at ...






\section{Literature}
\section{Literature}

This work is at the intersection of three literatures: economics on the study of speculative bubbles, network science on the prediction of outcomes based on features of an individual in a network, and a nacent field in the computer science security comunity that studies cryptocurrencies.


\subsection{Bubbles}

Perhaps the most striking line of research on bubbles in economics with respect to the cryptocurrency market is the study of those where the asset is worthless and this is common knowledge. 

In \cite{} " a bubble game that involves sequential trading of an
asset commonly known to be valueless. Because some traders do not
know where they stand in the market sequence, the game allows for
a bubble at the Nash equilibrium when there is no cap on the maximum
price. We run experiments both with and without a price cap.
Structural estimation of behavioral game theory models suggests that
quantal responses, uncertainty regarding other traders’ rationality,
and analogy-based expectations are important drivers of speculation. "


Decision-rule cascades and the dynamics of speculative bubbles
We combine Minsky’s Wnancial fragility analysis, behavioural analysis of decision rules and the
evolutionary economics of rule trajectories to provide an empirically grounded and computationally
tractable theory of the complex evolutionary dynamics of speculative Wnancial upswings. The behavioural
dynamics of asset bubbles can be conceptualized as the joint consequence of the adoption and
diVusion process of new investment decision rules coupled with the degradation of those rules as they
pass from a few expert investors to larger population of amateurs. We illustrate this using data covering
the recent Brisbane property market bubble (1999–2003) and show how it is consistent with the
existence of such cascading decision rules. We then explain how multi-agent simulation methods can
be used for modelling decision rule cascades.
© 2006 Elsevier B.V. All rights reserved."

\subsection{Prediction from networks}

"The Structural Virality of Online Diffusion" (Management Science)




"Here we propose a formal measure of what we label “structural virality” that interpolates between two conceptual extremes: content that gains its popularity through a single, large broadcast, and that which grows through multiple generations with any one individual directly responsible for only a fraction of the total adoption"

"We find that across all domains and all sizes of events, online diffusion is characterized by surprising structural diversity. Popular events, that is, regularly grow via both broadcast and viral mechanisms, as well as essentially all conceivable combinations of the two."

"we find that the correlation between the size of an event and its structural virality is surprisingly low, meaning that knowing how popular a piece of content is tells one little about how it spread"

"We find that while several of our empirical findings are consistent with such a model, it does not replicate the observed diversity of structural virality"











Network Diversity and Economic Development
http://www.sciencemag.org/content/328/5981/1029.full



"Hence, highly clustered, or insular, social ties are predicted to limit access to social and economic prospects from outside the social group, whereas heterogeneous social ties may generate these opportunities from a range of diverse contacts (1, 2)."

"Although both social and spatial network diversity scores were strongly correlated with IMD rank, we found a weaker positive correlation present using number of contacts and a negative correlation for communication volume."

"For example, whereas inhabitants of Stoke-on-Trent, one of the least prosperous regions in the UK, averaged a higher monthly call volume than the national average, they have one of the lowest diversity scores in the country. Similarly prosperous Stratford-upon-Avon has inhabitants with extremely diverse networks, despite no more communication than the national average. "




Predicting Spending Behavior Using Socio-mobile Features:
%http://ieeexplore.ieee.org/xpl/login.jsp?tp=&arnumber=6693330&url=http%3A%2F%2Fieeexplore.ieee.org%2Fxpls%2Fabs_all.jsp%3Farnumber%3D6693330
free version:
%https://scholar.google.com/citations?view_op=view_citation&hl=en&user=Ef1hJ8IAAAAJ&sortby=pubdate&citation_for_view=Ef1hJ8IAAAAJ:NaGl4SEjCO4C

Social behavior can be used to predict spending behavior in couples in regards to their prepensity to diversify the businesses they explore, become loyal customers and overspend. The results show that mobile phone social interaction patters can be more predictive than personality based features when predicting spending behavior. 

"We find that social behavior measured via face-to-face interaction, call, and SMS logs, can be used to predict the spending behavior for couples in terms of their propensity to explore diverse businesses, become loyal customers, and overspend"

"results show that mobile phone based social interaction patterns can provide more predictive power on spending behavior than personality based features. Interestingly, we find that more social couples also tend to overspend."




Money Walks: Implicit Mobility Behavior and Financial Well-Being:
%http://journals.plos.org/plosone/article?id=10.1371/journal.pone.0136628

Spatiotemporal traits such as exploration, engagement and elasticity can be used to predict future finanical difficulties. 

"Hence, in this work we study a large-scale cross-sectional dataset of human spending across space and time, and connect it to the biological phenomena of “foraging,” a basic pattern of animal movement to gather foods and resources."

"we analyzed a corpus of hundreds of thousands of human economic transactions and found that financial outcomes for individuals are intricately linked with their spatiotemporal traits like exploration, engagement, and elasticity. Such features yield models that are 30\% to 49\% better at predicting future financial difficulties than the comparable demographic models."

"As shown in Fig 2, individuals with lower levels of education (High School, Middle School, or Primary School) were found to be more likely to be late for their payments and get into financial trouble. Users with higher age were marginally less likely to overspend, miss payments, or get into financial trouble. Last, male customers and married customers were less likely to miss their payments."

"The figure also shows that multiple mobility behavior features were statistically correlated with outcome variables, even after controlling for the effect of abovementioned demographic variables of age, gender, marital status, education, and work type."

"the behavioral features were found to be more significantly associated (in terms of p-values) and contain higher predictive power (in terms of odds ratios being further away from 1.0 in either direction) as compared to the demographic features."

"The evidence so far indicating that each of the spatio-temporal behavioral descriptors has significant association with different financial outcomes motivates their combination to predict the financial outcome"




Predicting personality using novel mobile phone-based metrics
%http://dl.acm.org/citation.cfm?id=2456492
free version:
%https://www.google.com/url?sa=t&rct=j&q=&esrc=s&source=web&cd=1&ved=0CCEQFjAAahUKEwivqbnj0L3IAhWIcT4KHZtqCOg&url=http%3A%2F%2Frealitycommons.media.mit.edu%2Fdownload.php%3Ffile%3DdeMontjoye2013predicting-citation.pdf&usg=AFQjCNGDDMiemMv9vPQGBd_NNP30sEr6MQ&sig2=CQk3pVIzJ3EmXlV0lD8oKA

"Using a set of novel psychology-informed indicators that can be computed from data available to all carriers, we were able to predict users’ personality with a mean accuracy across traits of 42% better than random, reaching up to 61% accuracy on a three-class problem."

"The goal of the present research is to show that users’ personalities can be reliably inferred from basic information accessible from all mobile phones and to all service providers."


"The model predicted whether phone users were low, average, or high in neuroticism, extraversion, conscientiousness, agreeableness, and openness with an accuracy of 54%, 61%, 51%, 51%, and 49%, respectively."




\section{Data Description}
\subsection{Prices, Exchanges, and Coin characteristics}
\begin{figure}
\includegraphics[width=\columnwidth]{severity_volume}
\end{figure}
\textcolor{red}{Is this section needed here? Looks like it is exactly as what's described in variables section.}

Our main outcome measures are the severity of the largest drop in the price of each asset (cryptocoin), and the magnitude (volume) of the transactions made with the cryptocoin measured in USD.
We scrape daily price and volume data from coinmarketcap.com \footnote{\textcolor{red}{For robustness analysis smaller subsets of the coins where available from coin}}.
%JULIAN: Didn't follow the above footnote
We operationalize the severity of a bubble as the inverse of 1 dollar that would be lost buying at the maximum price and selling after that proportionally to the volume of the market until the present; we call this severity.
%JUliAN: Above is hard to follow if not spelled out in the form of equations; wordy papers with everything explained inline seems more economics and less WWW
We define the volume as the sum of the dollar volume of trade reported over all exchanges.

Many of the coins available in the exchanges are trivial modifications of an already existing coin in that they only differ in parameters such as the name, the number of total mineable coins, or the transaction time between blocks.
The production cost of these coins is virtually zero\footnote{There is a cottage industry that offers the creation of binaries and provisioning of mining and bandwidth as a bundled service that requires no technical skill from the user. Coingen or Coincreator are two examples of such services.}. To capture this effortless replication, we analyze two data sources:
\begin{enumerate}[topsep=0pt,itemsep=-0.5ex,partopsep=1ex,parsep=1ex]
 \item  \url{http://mapofcoins.com} which includes a genealogy of coins.
 \item Raw data from the github page of the coins not available in \url{http://mapofcoins.com}.
\end{enumerate}
If the coin to be analyzed has a parent and its algorithm differs from its parent or if it has no parent, it is labeled as nontrivial, indicating that the coin possesses legitimate technological innovation.


\subsection{Forum Discussions}
In order to study the effect of communication networks around cryptocoins on
price variations, we collected all the posts from the most popular cryptocurrency
online community, \url{https://bitcointalk.org}.  Our data consists of all the posts that were
made between January 2010 and July 2015 on the most active crypto-related forums:
\begin{enumerate}[topsep=0pt,itemsep=-0.5ex,partopsep=1ex,parsep=1ex]
  \item \textbf{Bitcoin Discussion:} This is the oldest forum on the website which mainly focuses
    on issues related only to Bitcoin. Interestingly, Satoshi Nakamoto, the alleged
    creator of Bitcoin made the first post on this forum in January 2010 and
    was active until January 2011. The presence of Satoshi in the data set enables us
    to study the position of various actors in the online community relative to Satoshi
    and its relation with the success or failure of cryptocoins they advocate or reject.
    %JULIAN: Why is people's relationship to Satoshi likely to be interesting or important?
  \item \textbf{Altcoin Discussion:} This is the most active forum in the community
    with more than 730,000 posts as of July 2015, dating back to June 2011.
    The discussions in this forum mainly evolve around alternative currencies
    other than Bitcoin. Users often discuss the merits or flaws of various
    altcoins or simply exchange technical information.
  \item \textbf{Announcement (Altcoin):} Community announcements such as development of 
    exchange clients or addition of new features are made here. This is an important forum
    in our study as the creation of new altcoins are announced here. Whenever a new
    altcoin is announced to the community, the announcement is tagged with string ANN.
    This enables us to detect announcements of new coins into the market and identify
    the users who introduced them for the first time.
  \item \textbf{Mining (Altcoin):} Technical issues pertaining to mining (i.e.~validating transactions)
    altcoins are discussed here.
  \item \textbf{Marketplace (Altcoin):} This forum contains the discussions on a wide-range of 
    market-related issues, such as price or volume trends, possible pump-and-dump schemes
    and exchange tips.
\end{enumerate}

Each forum consists of many subjects or threads initiated by different users.
Each thread contains several posts or replies, with an average of 10 posts per
thread.  The reply structure within each thread constitutes the basis of our
forum network, discussed below.  Each post has several fields which contain
valuable information in our context.
\begin{enumerate}[topsep=0pt,itemsep=-0.5ex,partopsep=1ex,parsep=1ex]
  \item \textbf{Subject:} Usually, the initiator of the thread chooses subject and all the
    following posts inherit the same subject.
  \item \textbf{Content:} The actual text of the post.
  \item \textbf{Position in the thread}: The later posts in the thread might not be as important
    as earlier posts and could be about issues other than the original topic of the thread.
  \item \textbf{Author}
  \item \textbf{Date}
\end{enumerate}
The community had only 10,000 unique users until early 2013, however it grew considerably faster after 2013 and reached about 70,000 by early 2015.
Nevertheless, there are only around 10,000 active users within any 30 day period on average.
Figure \ref{aurora_thread} shows an example of the first post in a thread which introduced Aurora coin for the first time in the Announcement forum.

\begin{figure}
\includegraphics[width=0.99\columnwidth]{aurora_thread_cut.pdf}
\caption{The first post of a thread announcing the release of Auroracoin for the first time in \url{https://bitcointalk.org} 
\label{aurora_thread}
\end{figure}

%JULIAN: Being crass, just showing a quick figure with an example of a discussion wouldn't hurt. I have no idea what people discuss on forums like this, so a figure whose purpose is to convince me that there's meaningful signal here could be helpful


\subsection{Forum Interaction Network}

Given the forum discussion data at the level of individual posts, we  construct a network capturing the discussion patterns among users. The structural properties of this network form the basis of our analysis on a per-coin basis. In this network, nodes are the forum users and \textit{directed edges} point from posters within each thread to thread-initiators. The omission of edges based on simple co-appearance within a thread leads to a sparser network which isolates the communication patterns around ``dialogue-shapers''. The edges in the discussion network are weighted by the number of times a poster replies to a thread-initiator in different threads (i.e.~multiple replies by the same user within the same thread are counted only once).
% Is this too big of a statement?
In this context, edge weights capture the level of engagement thread initiators receive from the community and the amount of information a poster receives from thread initiators. Furthermore, our network construction method uses all the interactions since the inception of bitcointalk in creating new edges or updating their weights. The unlimited retention of any such (replier to thread initiator) interaction captures relevant information on seniority and community influence which are obtained through long-term and persistent presence in the forums. 

%Prior to construction of the network, we merged posts from all forums into a
%single large forum since the community base of all five forums mentioned
%above is made of the same users and we are mostly concerned about influence and
%aggregate information flow among users, rather than the exact topic of the discussion.
To build our network, we first combine the posts from all forums. We do this because the community base of all five forums mentioned above is made of the same users, and because we are mostly concerned about influence and aggregate information flow among users, rather than the exact topic of the discussion.
The network construction involves replaying all the posts over time sorted by their date and updating the
discussion network accordingly. Whenever a new altcoin is introduced
in the forum for the first time, the user who introduced it and a snapshot of the network is taken. 
We analyze the discussion network only up to the first time each coin is introduced to the community, in order to avoid any possible confounding between a coin's price movement and the extra attention it receives in the community due the same price changes. Our method uses the position of the first introducer in the network snapshot and the general structure of her neighborhood for extracting various network measures corresponding to that coin. Our final analysis examines these per-coin measures for evaluating the performance of each coin.



%TODO Do we use the date of the ANN or the date of the first trade?!?! NIKETE & EAMAN.
% E: Both of them are in the file, depends which one you are using. 
%The identification of true introductions of new altcoins is a difficult process prone to many false-positives. 
The  majority of such introductions are made in the \textit{Announcement} forum and are preceded with the ``ANN'' tag. We look for the first mention of both the coin symbol \textbf{and} its descriptive name in the subject of a thread which contains the announcement tag. The first mentions of either the coin symbol \textbf{or} its name are used as a fall-back in case the more restrictive \textbf{and} requirement did not detect the first introduction of the coin.
%JULIAN: The above seems like a long winded way of saying that you just used "or". i.e., the above sentence parses to "(a and b) or (a or b)". Didn't want to overwrite in case I misunderstood.
% EAMAN: remove the text on OR requirement, if you did not or do not plan on falling back on OR mentions.
Using the more restrictive matching requiring both the coin name \textbf{and} symbol be present in the subject, we are able to detect the first introduction of 376 altcoins out of 679. 
We can detect an extra 176 altcoins by falling back on the \textbf{or} requirement.
% Eaman: remove text in red if not falling back on OR.
The forum user who initiated such a thread is assigned as the introducer of the coin to the community.
Aproximately 500 of the 600 coins where manualy verfied in the forums by two of the authors to have correct identification.
%TODO(NIKETE): add validation results table wrt mapofcoins data here 



\section{Variables}
\subsection{Magnitude and Severity} \label{variables_nikete}
Our main outcome measures are the severity of the largest drop in the price of each cryptocoin, and the magnitude (volume) of the transactions made with the cryptocoin measured in USD.
We operationalize the intensity of a bubble as the proportion of a 1 dollar that would be lost buying at the maximum price and selling after that proportionally to the volume of the market till the present. Magnitude of a bubble is defined as the proportion of the average daily volume prior to maximum price to the overall average daily volume.
More formally for each coin $c$, let $P_c,V_c$ be two vectors of length $T$ each indexed by days since the coin $c$ started trading.
Let $v_{c,t}$, the $t^{th}$ element of $V_c$ be the number of $c$ coins traded across all exchanges $t$ days after $c$'s introduction to the market. 
Similarly, let $p_t$ be the representative price of coin $c$ on day $t$.
\footnote{There are several ways that this can be defined reasonably. For the markets organized as continuous double auctions the price of the first or last transaction of the day,  the average price transacted during a day, the average best asking bid at midnight or noon, would all be reasonable choices. Exchange price aggregators and exchanges own historical data do not provide enough precision to pin this down, as it is entirely possible that the aggregators are using inconsistent definitions for the underlying exchanges.}. Then, we simply define our magnitude measure for coin $c$ as
\begin{equation}
magnitude_{c} = \log(\frac{\sum_{t=1}^{t_{max}} v_t p_t} {\sum_{t=1}^{T} v_t  p_t} \frac{T}{t_{max}})
\end{equation}
where $t_{max}$ is the date in which the coin realized its maximum price and T is the number of days since the introduction of coin $c$ to the market until present ($t_{max} < T$). Note that the magnitude is normalized by the length of time coin has been present in the market.

Similarly, the average price after $t_{max}$ weighted by daily volume is defined as:
\begin{equation}
\bar{p}_{t_{max},T} = \frac{\sum_{t=t_{max}}^{T} v_t p_t} {\sum_{t=t_{max}}^{T} v_t}
\end{equation}

We define the severity of the bubble experienced by the coin $c$ as the fraction of the maximum price lost relative to average price since $t_{max}$ until present:
\begin{equation}
severity_{c} = \log(\frac{ p_{t_{max}}} {\bar{p}_{(t_{max},T)} })
\end{equation}

Figure \ref{log_magnitude_log_severity} shows the distribution of our constructed dependent variables and their relationship. Both of the variable exhibit a heavy-tailed distribution which makes it necessary to use a robust regression method.

\begin{figure}
\centering
\includegraphics[width=\columnwidth]{log_magnitude_log_severity.pdf}
\caption{The relative distribution of our dependent variables. Note that the magnitude is shown in terms of its natural log due to its heavy-tailed distribution. Both outcome measures exhibit a heavy-tailed non-normal distribution which is a violation of simple OLS assumptions. For this reason, our analysis uses a robust regression method using Huber weights.}
\label{log_magnitude_log_severity}
\end{figure}

\subsection{Network Structure}
\subsection{Network Structure}
In this section, we discuss the various metrics extracted from discussion networks and used as independent variables in the regression analysis. Many of these variables are standard metrics in graph theory designed to capture node centrality is specific scenarios \cite{KleinbergNetworks}. As mentioned before, each coin is associated with a forum user and a discussion network which corresponds to the state of the forum at the time the user introduced the coin to the community. All of our node-level variables refer to the user introducing the coin. Below, we list the network variables included in the analysis. We used Python igraph implementation for computing the network-related metrics \cite{igraph}.
\begin{enumerate}[topsep=0pt,itemsep=-0.5ex,partopsep=1ex,parsep=1ex]
  \item \textbf{Introducer number of posts:} The total number of posts (thread-initiations or simple replies) the coin introducer has made at the time she introduces the coin. It captures the user's level of activity in the community.
  \item \textbf{Introducer number of threads:} The total number of threads the coin introducer has made. Users who start many threads are more likely to receive incoming edges and to shape the dialogue in the community.
  \item \textbf{Seniority:} It is the number of days since the user's first post in the forums. We use this as a proxy for user's seniority in the community.
  \item \textbf{Incoming degree:} The (incoming) degree centrality captures the role of dialogue-shapers in the community as it is the number of unique users who have replied to any of the focal user's threads.
  \item \textbf{Outgoing degree:} The (outgoing) degree centrality captures the role followers in the community as it is the number of unique thread initiators the focal user has ever replied to. 
  \item \textbf{Total degree:} The (undirected) degree centrality captures the total level of the user's involvement in the community in any of the two forms above.
  \item \textbf{Clustering Coefficient:} A measure of embeddedness or triadic closure, this is the fraction of the focal user's triads that are closed. In general, ideas are more likely to be reinforced and persistent in a triad if it is `tightly-knit'. Such a positive effect of triadic closure (and balance) on tie qualities and their persistence is shown to exist in online social network such as Twitter \cite{KleinbergBalance}, and we believe the same argument applies to our scenario.
  %JULIAN: To be honest I'd do away with most of the above definitions for the social networks track, and just quickly list the features. A quick explanation of why they ought to be useful would be valuable, but definitions for these basic concepts is overkill.
  \item \textbf{Unweighted closeness centrality:} While degree centrality measures the level of user engagement in the community, it only examines the local structure around the user. It's possible for the focal user to have a high degree centrality, but only in an isolated subcommunity. In contrast, closeness centrality measures the level of the user's engagement with the global network \textit{either directly or indirectly}. It is relevant in many scenarios, including in online discussions, as information spreads via shortest paths. 
  
  In our context, a user with high incoming closeness centrality has initiated many threads and received replies from a diverse set of users who themselves are close to a large set of diverse users. Similarly, a user with high outgoing closeness centrality has replied to a diverse set of thread-initiators who themselves are close to a large set of diverse users. Our analysis consisted of three versions of the unweighted closeness centrality:
  \begin{enumerate}
    \item \textbf{Incoming:} Only the directed paths leading to the focal user are used. In other words, it measures closeness of the whole network to the focal user. Users who start many threads are likely to have higher incoming closeness centrality.
    \item \textbf{Outgoing:} Only the directed paths starting from the focal user to all the other users are used. In other words, it measures closeness of the focal user to the whole network. Users who reply to many threads are likely to have higher outgoing closeness centrality.
    \item \textbf{Undirected:} The paths both from and to the focal users are used. Users who initiate and reply to many threads are likely to have higher undirected closeness centrality.
  \end{list} 
  All measures are normalized by the number of users present in the network at the time of coin introduction, so that the comparison between the closeness centrality of various users (who introduce the coins) at different times  is valid. 
  \item \textbf{Weighted closeness centrality:} Similar to the unweighted closeness centrality above, we computed three different versions, with the exception that edges are weighted to indicate the distance or level of interaction intensity between the two users. The edges weights in our discussion network are determined by the frequency of interactions between two users; and as two users interact more, they are deemed to be closer in their shortest path. Thus in the computation of weighted closeness centralities, we use the reciprocal of the weights as the distance between two users.
  \begin{equation}
    C_{i} = (N-1)\sum_{j=1}^{N}\sum_{e \in S_{ij}} w_{e} 
  \end{equation}
  where $e$ denotes an edge in $S_{ij}$ the set of users on the shortest path from i to j. $w_e$ is the weight of edge $e$ determined by the number of interactions between the end points.
  
  \item \textbf{Weighted betweenness centrality:} Betweenness measures how many shortest paths have to go through the focal user in order for the end users to reach one another. It is closely related to the theory of weak ties and structural holes and measures how well of a bridge is the focal user. In our context, one could interpret betweenness centrality as a generational bridge. Bitcointalk has been an active forum since early 2010, and many users who were once active in its early days are no longer present in the forum. There are however some early users who are still active on the forum. These users have high betweenness centrality as they act as generational bridges between founders and the newcomers to the community. Another standard interpretation of high betweenness centrality is the existence of users who simulatenously interact with two isolated communities in the forum. Similar to closeness centrality, our betweenness centrality computation uses the inverse of edge weights as the distance between two users.
  \item \textbf{satoshi distance:} 
  \item \textbf{satoshi pagerank weighted:} 
  \item \textbf{pagerank weighted:} 
\end{enumerate}


\section{Methods }

\section{Methods }

Initially we we start with a baseline model that considers only the user characteristics that are easily observable from their activity on the forum before the announcement: the number of posts and of subjects,  the time since they first post, the number of users that they have responded to and received responses from. 
These network measures are possible for any generic discussion, we introduce two further sets of variables to enrich our models that rely on domain knowledge of the underlying assets: Satoshi network measures, and weather a given coin is embodied in new software or if it is simply a change in name and parameters of the codebase used by a different coin.

We estimate linear regularized least squares (ElasticNet cite TODO) using a combination of L1 and L2 norm, with their parameters set by 5 fold cross validation. 
We then estimate a OLS model of the support of the variables and calculate White robust standard errors, to allow for model introspection. 
Disclaimer that the regularization might make them not match (TODO: add set with normal SE that is estimated with the regularization, in results compare the coefficients) 
To evaluate nonlinearities and interactions  in the model we fit a gradient boosted machine on the full support, cross validating its hyper parameters; as well as on the OLS selected subset.  TODO add graphs showing interactions and nonlinearities; table with model comparisons.


The initial analysis pipeline and debugging, hyperparameter setting was done using only th initial 270 of the eventual 560 in the sample. The full set of samples used for these estimates was only estimated before writing the results section. The method will not be revised beyond this point.



\section{Results}
We fitted $7$ models with our methodology for each of our two outcome variables constructed based on the first user who used either the coin name or symbol in an ANN tagged thread.
Tables 1 and 2 report the coefficient estimates for standardized variables (both outcomes and measures have been linearly transformed to have mean 0 and standard deviation 1).
As a robustness check in the appendix tables 3 and 4 show the same results when applied to a 

Our User model considers how long a user has been active, and how many users have replied to them and he has replied to. 
Our nontrivial model consists of a single binary variable encoding wether the source code of the coin is a not a trivial change to existing coins. 
Our Satoshi model considers the distance and pagerank relative to satoshi.
None  of these three models achieve out of sample error rates beyond simply predicting $0$ constantly for either of our outcome measures of interest.


Our Network model encodes directed edges when a user interacts with another, and proceeds to estimate closeness centrality and clustering measures based on this. 
Our Weighted model uses the intensity of interactions to create a weighted network and construct a similar set of measures to the Network model.
Our interaction model includes interaction terms between non-trivialness and the weighted network measures.
Our All model contains all previous mentioned variables, there is extreme collinearity in this model and thus it's estimated coefficients and standard errors should be interpreted with extreme caution. 


\begin{figure}[h]
\centering
\begin{subfigure}
{\includegraphics[width=\columnwidth]{centrality_volume_severity.pdf}}
\end{subfigure}
\begin{subfigure}
{\includegraphics[width=\columnwidth]{centrality_volume_severity_nontrivial}}
\end{subfigure}
\caption{Interaction of Severity and Magnitude metrics with introducer closeness centrality (top), segmented by nontrivial classification (below). \textcolor{red}{colorbar label needs to change to introducer closeness centrality, switch bottm plot to pdf version. But for some reason left subplot contours don't show up.}}
\label{centrality_volume_severity}
\end{figure}

The main driver of our explanatory power is the centrality of a user in the directed network derived from the forum. 
This effect appears to be mediated by whether a coin involves a nontrivial technological change, the direction of the interaction reversing depending on whether it relates to magnitude or severity.
While the linear model is not able to fully capture this interaction, plotting the values of severity and magnitude of bubbles and the centrality reveals a interaction; with the polarity of the gradients flipping in the severity case.

%JULIAN: Above sentence is tough to parse, but kind of an important one.
Both the severity and the magnitude of bubbles increases with the centrality of the user who introduces the coins in the forums.  
Interestingly this effect is concentrated in different ways depending on whether the coin software is more than a trivial modification: trivial coins have more severe bubbles the more central their introducers are, while volume is greater the more central the introducer of a nontrivial coin is.






\begin{figure}[h]
\centering
\begin{subfigure}
{\includegraphics[width=\columnwidth]{cluster_closeness_volume.pdf}}
\vspace{-2\baselineskip}
\end{subfigure}
\begin{subfigure}
{\includegraphics[width=\columnwidth]{cluster_closeness_severity.pdf}}
\end{subfigure}
\caption{Interaction of introducer closeness centrality and clustering coefficient with bubble Magnitude (top) and Severity (bottom). \textcolor{red}{colorbar label needs to change to Magnitude and Severity.}}
\label{cluster_closeness_volume}
\end{figure}


\begin{table*}
\centering
\caption{Severity OR}
\begin{center}
\begin{tabular}{lccccccc}
\hline
                                                 & Model    & Nontrivial & Satoshi & Network & Weighted & Interaction &   All    \\
\hline
\hline

Intercept                                        & 0.00     & 0.00       & 0.00    & 0.00    & 0.00     & 0.00               & 0.00          \\
                                                 & (0.00)   & (0.00)     & (0.00)  & (0.00)  & (0.00)   & (0.00)             & (0.04)        \\
nontrivial                                       &          &            &         &         &          & 0.00               & 0.00          \\
                                                 &          &            &         &         &          & (0.00)             & (0.04)        \\
user1 betweenness centrality weighted            &          &            &         &         & 0.01     &                    & -0.04         \\
                                                 &          &            &         &         & (0.07)   &                    & (0.18)        \\
user1 betweenness centrality weighted:nontrivial &          &            &         &         &          & 0.00               &               \\
                                                 &          &            &         &         &          & (0.00)             &               \\
user1 closeness centrality unweighted            &          &            &         & 0.12*** &          & 0.16***            & 1.69          \\
                                                 &          &            &         & (0.04)  &          & (0.05)             & (3.64)        \\
user1 closeness centrality weighted              &          &            &         &         & 0.15***  &                    & -1.60         \\
                                                 &          &            &         &         & (0.04)   &                    & (3.62)        \\
user1 closeness centrality weighted:nontrivial   &          &            &         &         &          & 0.00               &               \\
                                                 &          &            &         &         &          & (0.00)             &               \\
user1 clustering coefficient                     &          &            &         & 0.00    &          & 0.00               & 0.07          \\
                                                 &          &            &         & (0.00)  &          & (0.00)             & (0.05)        \\
user1 clustering coefficient:nontrivial          &          &            &         &         &          & 0.00               &               \\
                                                 &          &            &         &         &          & (0.00)             &               \\
user1 days since first post                      & 0.00     &            &         & 0.00    & -0.03    & 0.00               & 0.02          \\
                                                 & (0.00)   &            &         & (0.00)  & (0.05)   & (0.00)             & (0.06)        \\
user1 degree incoming                            & 0.03     &            &         & 0.02    &          & 0.04               & -0.06         \\
                                                 & (0.04)   &            &         & (0.04)  &          & (0.04)             & (0.06)        \\
user1 degree outgoing                            &          &            &         &         &          &                    & 0.04          \\
                                                 &          &            &         &         &          &                    & (0.10)        \\
user1 degree total                               & 0.00     &            &         & 0.00    &          & 0.00               & -0.14**       \\
                                                 & (0.00)   &            &         & (0.00)  &          & (0.00)             & (0.06)        \\
user1 num posts                                  & 0.00     &            &         & 0.00    & -0.04    & 0.00               & -0.14         \\
                                                 & (0.00)   &            &         & (0.00)  & (0.08)   & (0.00)             & (0.11)        \\
user1 num subjects                               &          &            &         &         & -0.06    &                    & 0.00          \\
                                                 &          &            &         &         & (0.06)   &                    & (0.09)        \\
user1 pagerank weighted                          &          &            &         &         & 0.24***  &                    & 0.93***       \\
                                                 &          &            &         &         & (0.05)   &                    & (0.17)        \\
user1 satoshi distance                           &          &            &         &         &          &                    & -0.34***      \\
                                                 &          &            &         &         &          &                    & (0.11)        \\
user1 satoshi distance inf                       &          &            &         &         &          &                    & 0.30***       \\
                                                 &          &            &         &         &          &                    & (0.09)        \\
user1 satoshi pagerank weighted                  &          &            &         &         &          &                    & -0.64***      \\
                                                 &          &            &         &         &          &                    & (0.16)        \\
R2                                               & 0.01     & 0.00       & 0.00    & 0.04    & 0.10     & 0.07               & 0.20          \\
ElasticNet CV MSE:                               & 1.00     & 1.00       & 1.02    & 0.98    & 0.95     & 0.97               & 0.91          \\
BIC                                              & 1594     & 1572       & 1572    & 1585    & 1551     & 1340               & 1292          \\
N                                                & 552      & 552        & 552     & 552     & 552      & 460                & 460           \\
Adjusted-R2                                      & -0.00    & 0.00       & 0.00    & 0.03    & 0.09     & 0.04               & 0.18          \\
Condition Number                                 & 12.86    & 1.00       & 1.00    & 12.96   & 4.17     & 15.97              & 360428331.38  \\
\hline
\end{tabular}
\end{center}
\end{table*}

\begin{table*}
\centering
\caption{Volume OR}
\begin{center}
\begin{tabular}{lccccccc}
\hline
                                               & Model    & Nontrivial & Satoshi & Network & Weighted & Interaction &  All    \\
\hline
\hline

Intercept                                      & 0.00     & 0.05       & 0.08*   & 0.00    & 0.07     & 0.07               & 0.04    \\
                                               & (0.00)   & (0.05)     & (0.05)  & (0.00)  & (0.04)   & (0.04)             & (0.04)  \\
nontrivial                                     &          & 0.11**     & 0.14*** & 0.05    & 0.12***  & 0.11**             & 0.10**  \\
                                               &          & (0.05)     & (0.05)  & (0.04)  & (0.04)   & (0.04)             & (0.04)  \\
user1 closeness centrality unweighted          &          &            &         & 0.16*** &          & 0.25***            & 0.16    \\
                                               &          &            &         & (0.05)  &          & (0.05)             & (3.50)  \\
user1 closeness centrality weighted            &          &            &         &         & 0.23***  &                    & 0.07    \\
                                               &          &            &         &         & (0.05)   &                    & (3.49)  \\
user1 closeness centrality weighted:nontrivial &          &            &         &         &          & 0.06               &         \\
                                               &          &            &         &         &          & (0.04)             &         \\
user1 clustering coefficient                   &          &            &         & 0.00    &          & -0.07              & -0.04   \\
                                               &          &            &         & (0.00)  &          & (0.05)             & (0.05)  \\
user1 days since first post                    &          &            & 0.08    & 0.00    & 0.05     & 0.06               & 0.02    \\
                                               &          &            & (0.05)  & (0.00)  & (0.05)   & (0.05)             & (0.05)  \\
user1 degree incoming                          &          &            &         & 0.00    &          &                    &         \\
                                               &          &            &         & (0.00)  &          &                    &         \\
user1 degree outgoing                          &          &            &         & 0.00    & 0.00     & 0.00               & 0.00    \\
                                               &          &            &         & (0.00)  & (0.00)   & (0.00)             & (0.00)  \\
user1 degree total                             &          &            &         & 0.00    &          & -0.01              &         \\
                                               &          &            &         & (0.00)  &          & (0.05)             &         \\
user1 num posts                                &          &            &         & 0.00    & 0.00     &                    & 0.00    \\
                                               &          &            &         & (0.00)  & (0.00)   &                    & (0.00)  \\
user1 num subjects                             &          &            & 0.02    &         &          &                    &         \\
                                               &          &            & (0.06)  &         &          &                    &         \\
user1 pagerank weighted                        &          &            &         &         & -0.00    &                    & 0.00    \\
                                               &          &            &         &         & (0.04)   &                    & (0.00)  \\
user1 satoshi distance                         &          &            &         &         &          &                    & 0.00    \\
                                               &          &            &         &         &          &                    & (0.00)  \\
R2                                             & -0.00    & 0.02       & 0.04    & 0.07    & 0.10     & 0.12               & 0.10    \\
ElasticNet CV MSE:                             & 1.03     & 0.99       & 0.99    & 0.94    & 0.96     & 0.95               & 0.95    \\
BIC                                            & 1575     & 1292       & 1297    & 1308    & 1286     & 1283               & 1305    \\
N                                              & 553      & 460        & 460     & 460     & 460      & 460                & 460     \\
Adjusted-R2                                    & -0.00    & 0.02       & 0.03    & 0.05    & 0.09     & 0.10               & 0.08    \\
Condition Number                               & 1.00     & 1.00       & 1.34    & nan     & 3.67     & 3.15               & 200.37  \\
\hline
\end{tabular}
\end{center}
\end{table*}


\subsection{Future Work}
Our results suggest that bubble dynamics may be strongly influenced by founder effects, but that traditional network measures based on their position in the aggregate discussion graph do not provide a tight characterization of this effect.
As future work we are looking at coarse (positive/negative, detailed/cursory) semantic analysis of the discussion, and evidence of prior cooperation between pairs of participants in other altcoin markets, in order to attempt a more accurate characterization of core participants and their actions.

The features of the node as well as the construction of the graph are informed by a the pre-existing literature which does not allow for much complexity on the edges of the graphs beyond weight and direction.
A promising avenue for future work is to explore richer models for edges, in particular allowing them to exist in latent spaces, where he influence or attnetion that is payed by responding to a user is colored by the wording of the response. 
Two different avenues to learn such model suggest themselves: either using higher resolution time dynamics of the prices to learn to learn a space that captures the expectations implicitly forecasted by different language, or using NLP to attempt to parse these using other external corporate to know the multidimensional valence of the words.
%JULIAN: Seems a little too vague above
The cross-sectional design with time separation does not allow us to take advantage of intra-coin variation.
Beyond using data from forums, code repositories could also be exploited in future work as a rich source of variation beyond the binary feature of whether a coin is a trivial fork or not.


\section{Conclusion}
\section{Conclusion}


The total variance accounted for is small, so you need a discussion like "results suggest that bubble dynamics may be strongly influenced by a core set of participants, but that traditional network measures on the aggregate discussion graph do not provide a tight characterization of this core group.  We are looking at coarse (positive/negative, detailed/cursory) semantic analysis of the discussion, and evidence of prior cooperation between pairs of participants in other altcoin markets, in order to attempt this more accurate characterization of core participants and their actions.


%\begin{table*}
\caption{Log(Volume) model OLS parameter estimates with heteroskedasticity robust SE after ElasticNet variable selection}
\begin{center}
\begin{tabular}{lccccccc}
\hline
                                                 & Activity & Nontrivial & Satoshi & Network &   Weighted   & Network*Nontrivial &   All     \\
\hline
\hline
\end{tabular}
\begin{tabular}{llllllll}
Intercept                                        & 0.00     & 0.00       & 0.00    & 0.00    & 0.00         & 0.00               & 0.00      \\
                                                 & (0.00)   & (0.00)     & (0.00)  & (0.00)  & (0.00)       & (0.00)             & (0.00)    \\
nontrivial                                       &          & 0.15***    & 0.13**  & 0.06    & 0.06         & 0.04               & 0.10**    \\
                                                 &          & (0.05)     & (0.05)  & (0.05)  & (0.05)       & (0.05)             & (0.05)    \\
 closeness centrality unweighted            &          &            &         & 0.24*** &              & 0.21***            & 0.00      \\
                                                 &          &            &         & (0.06)  &              & (0.06)             & (0.00)    \\
 closeness centrality unweighted:nontrivial &          &            &         &         &              & 0.06               &           \\
                                                 &          &            &         &         &              & (0.04)             &           \\
 closeness centrality weighted              &          &            &         &         & 0.22***      &                    & 0.39***   \\
                                                 &          &            &         &         & (0.05)       &                    & (0.06)    \\
 clustering coefficient                     &          &            &         & -0.04   &              & -0.03              & -0.18***  \\
                                                 &          &            &         & (0.06)  &              & (0.06)             & (0.06)    \\
 days since first post                      &          &            & 0.06    & 0.01    & 0.01         & 0.01               & 0.09      \\
                                                 &          &            & (0.06)  & (0.05)  & (0.05)       & (0.05)             & (0.06)    \\
 degree incoming                            &          &            & 0.04    & 0.00    & 0.00         & 0.00               & 0.00      \\
                                                 &          &            & (0.06)  & (0.00)  & (0.00)       & (0.00)             & (0.00)    \\
 degree outgoing                            &          &            &         & 0.00    & 0.00         & 0.00               & 0.07      \\
                                                 &          &            &         & (0.00)  & (0.00)       & (0.00)             & (0.11)    \\
 degree total                               &          &            &         & 0.00    & 0.00         & 0.00               & 0.05      \\
                                                 &          &            &         & (0.00)  & (0.00)       & (0.00)             & (0.10)    \\
 num posts                                  &          &            & -0.03   & 0.00    & 0.00         & 0.00               & -0.17     \\
                                                 &          &            & (0.06)  & (0.00)  & (0.00)       & (0.00)             & (0.10)    \\
 num subjects                               &          &            &         & 0.00    & 0.00         & 0.00               & -0.05     \\
                                                 &          &            &         & (0.00)  & (0.00)       & (0.00)             & (0.06)    \\
 pagerank weighted                          &          &            &         &         & 0.00         &                    & -0.17     \\
                                                 &          &            &         &         & (0.00)       &                    & (0.19)    \\
 satoshi distance                           &          &            &         &         &              &                    & 0.00      \\
                                                 &          &            &         &         &              &                    & (0.00)    \\
 satoshi distance inf                       &          &            &         &         &              &                    & -0.04     \\
                                                 &          &            &         &         &              &                    & (0.05)    \\
 satoshi pagerank weighted                  &          &            &         &         &              &                    & 0.18      \\
                                                 &          &            &         &         &              &                    & (0.19)    \\
R2                                               & 0.00     & 0.02       & 0.04    & 0.10    & 0.10         & 0.11               & 0.16      \\
ElasticNet CV MSE:                               & 1.01     & 0.99       & 0.99    & 0.94    & 0.95         & 0.94               & 0.94      \\
BIC                                              & 1072     & 1069       & 1082    & 1078    & 1082         & 1080               & 1084      \\
N                                                & 376      & 376        & 376     & 376     & 376          & 376                & 376       \\
Adjusted-R2                                      & 0.00     & 0.02       & 0.03    & 0.09    & 0.08         & 0.09               & 0.13      \\
Condition Number                                 & 1.00     & 1.00       & 2.05    & nan     & 147295588.13 & nan                & nan       \\
\hline
\end{tabular}
\end{center}
\end{table*}

%\begin{table*}
\caption{Log(Severity) model OLS parameter estimates with heteroskedasticity robust SE after ElasticNet variable selection }
\begin{center}
\begin{tabular}{lccccccc}
\hline
                                                 & Activity & Nontrivial & Satoshi & Network &   Weighted   & Network*Nontrivial &   All    \\
\hline
\hline
\end{tabular}
\begin{tabular}{llllllll}
Intercept                                        & 0.00     & 0.00       & 0.00    & 0.00    & 0.00         & 0.00               & 0.00     \\
                                                 & (0.00)   & (0.00)     & (0.00)  & (0.00)  & (0.00)       & (0.00)             & (0.00)   \\
nontrivial                                       &          &            &         &         & -0.03        &                    &          \\
                                                 &          &            &         &         & (0.05)       &                    &          \\
closeness centrality unweighted            &          &            &         & 0.00    &              & 0.00               & 0.15**   \\
                                                 &          &            &         & (0.00)  &              & (0.00)             & (0.06)   \\
closeness centrality unweighted:nontrivial &          &            &         &         &              & 0.00               &          \\
                                                 &          &            &         &         &              & (0.00)             &          \\
closeness centrality weighted              &          &            &         &         & 0.30***      &                    & 0.00     \\
                                                 &          &            &         &         & (0.05)       &                    & (0.00)   \\
clustering coefficient                     &          &            &         & 0.00    &              & 0.00               & 0.23***  \\
                                                 &          &            &         & (0.00)  &              & (0.00)             & (0.05)   \\
days since first post                      &          &            &         &         & -0.02        &                    & -0.01    \\
                                                 &          &            &         &         & (0.06)       &                    & (0.06)   \\
degree incoming                            &          &            &         & 0.00    & -0.09*       & 0.00               & -0.09    \\
                                                 &          &            &         & (0.00)  & (0.05)       & (0.00)             & (0.07)   \\
degree outgoing                            &          &            &         &         & 0.09         &                    & 0.00     \\
                                                 &          &            &         &         & (0.09)       &                    & (0.00)   \\
degree total                               &          &            &         &         & -0.04        &                    & 0.00     \\
                                                 &          &            &         &         & (0.03)       &                    & (0.00)   \\
num posts                                  &          &            &         &         & -0.16        &                    & -0.08    \\
                                                 &          &            &         &         & (0.10)       &                    & (0.07)   \\
num subjects                               &          &            &         &         & -0.00        &                    & -0.01    \\
                                                 &          &            &         &         & (0.06)       &                    & (0.06)   \\
pagerank weighted                          &          &            &         &         & 0.36***      &                    & 0.55***  \\
                                                 &          &            &         &         & (0.07)       &                    & (0.18)   \\
satoshi distance                           &          &            &         &         &              &                    & -0.11    \\
                                                 &          &            &         &         &              &                    & (0.11)   \\
satoshi distance inf                       &          &            &         &         &              &                    & 0.08     \\
                                                 &          &            &         &         &              &                    & (0.10)   \\
satoshi pagerank weighted                  &          &            &         &         &              &                    & -0.22    \\
                                                 &          &            &         &         &              &                    & (0.17)   \\
R2                                               & 0.00     & 0.00       & 0.00    & 0.00    & 0.22         & 0.00               & 0.27     \\
ElasticNet CV MSE:                               & 1.01     & 1.01       & 1.01    & 0.94    & 0.91         & 0.94               & 0.90     \\
BIC                                              & 1072     & 1072       & 1072    & 1090    & 1028         & 1096               & 1024     \\
N                                                & 376      & 376        & 376     & 376     & 376          & 376                & 376      \\
Adjusted-R2                                      & 0.00     & 0.00       & 0.00    & -0.01   & 0.20         & -0.01              & 0.25     \\
Condition Number                                 & 1.00     & 1.00       & 1.00    & 1.80    & 147295588.13 & 2.17               & nan      \\
\hline
\end{tabular}
\end{center}
\end{table*}





%\section{Acknowledgments}

%
% The following two commands are all you need in the
% initial runs of your .tex file to
% produce the bibliography for the citations in your paper.
\bibliographystyle{abbrv}
\bibliography{bubble}  % sigproc.bib is the name of the Bibliography in this case
% You must have a proper ".bib" file
%  and remember to run:
% latex bibtex latex latex
% to resolve all references
%
% ACM needs 'a single self-contained file'!
%
%APPENDICES are optional
%\balancecolumns
\appendix
Appendix A
\begin{table*}
\caption{Volume AND}
\begin{center}
\begin{tabular}{lccccccc}
\hline
                                               & Activity & Nontrivial & Satoshi & Network & Weighted & Network*Nontrivial &   All     \\
\hline
\hline
\end{tabular}
\begin{tabular}{llllllll}
Intercept                                      & 0.00     & 0.00       & 0.00    & 0.00    & 0.00     & 0.00               & 0.00      \\
                                               & (0.00)   & (0.00)     & (0.00)  & (0.00)  & (0.00)   & (0.00)             & (0.00)    \\
nontrivial                                     &          & 0.15***    & 0.13**  & 0.06    & 0.04     & 0.02               & 0.11**    \\
                                               &          & (0.05)     & (0.05)  & (0.05)  & (0.05)   & (0.05)             & (0.05)    \\
user1 betweenness centrality weighted          &          &            &         &         & 0.00     &                    & -0.04     \\
                                               &          &            &         &         & (0.00)   &                    & (0.08)    \\
user1 closeness centrality unweighted          &          &            &         & 0.24*** &          & 0.18***            & 0.00      \\
                                               &          &            &         & (0.06)  &          & (0.05)             & (0.00)    \\
user1 closeness centrality weighted            &          &            &         &         & 0.19***  &                    & 0.39***   \\
                                               &          &            &         &         & (0.05)   &                    & (0.07)    \\
user1 closeness centrality weighted:nontrivial &          &            &         &         &          & 0.06               &           \\
                                               &          &            &         &         &          & (0.04)             &           \\
user1 clustering coefficient                   &          &            &         & -0.04   &          & 0.00               & -0.18***  \\
                                               &          &            &         & (0.06)  &          & (0.00)             & (0.06)    \\
user1 clustering coefficient:nontrivial        &          &            &         &         &          & 0.00               &           \\
                                               &          &            &         &         &          & (0.00)             &           \\
user1 days since first post                    &          &            & 0.06    & 0.01    & 0.00     & 0.00               & 0.09      \\
                                               &          &            & (0.06)  & (0.05)  & (0.00)   & (0.00)             & (0.06)    \\
user1 degree incoming                          &          &            & 0.04    & 0.00    & 0.00     & 0.00               & 0.00      \\
                                               &          &            & (0.06)  & (0.00)  & (0.00)   & (0.00)             & (0.00)    \\
user1 degree outgoing                          &          &            &         & 0.00    & 0.00     & 0.00               & 0.09      \\
                                               &          &            &         & (0.00)  & (0.00)   & (0.00)             & (0.11)    \\
user1 degree total                             &          &            &         & 0.00    & 0.00     & 0.00               & 0.08      \\
                                               &          &            &         & (0.00)  & (0.00)   & (0.00)             & (0.12)    \\
user1 num posts                                &          &            & -0.03   & 0.00    & 0.00     & 0.00               & -0.17*    \\
                                               &          &            & (0.06)  & (0.00)  & (0.00)   & (0.00)             & (0.10)    \\
user1 num subjects                             &          &            &         & 0.00    & 0.00     & 0.00               & -0.05     \\
                                               &          &            &         & (0.00)  & (0.00)   & (0.00)             & (0.06)    \\
user1 pagerank weighted                        &          &            &         &         & 0.00     &                    & -0.17     \\
                                               &          &            &         &         & (0.00)   &                    & (0.19)    \\
user1 satoshi distance                         &          &            &         &         &          &                    & 0.00      \\
                                               &          &            &         &         &          &                    & (0.11)    \\
user1 satoshi distance inf                     &          &            &         &         &          &                    & -0.04     \\
                                               &          &            &         &         &          &                    & (0.10)    \\
user1 satoshi pagerank weighted                &          &            &         &         &          &                    & 0.17      \\
                                               &          &            &         &         &          &                    & (0.19)    \\
R2                                             & 0.00     & 0.02       & 0.04    & 0.10    & 0.09     & 0.10               & 0.16      \\
ElasticNet CV MSE:                             & 1.01     & 0.99       & 0.99    & 0.94    & 0.95     & 0.95               & 0.94      \\
BIC                                            & 1072     & 1069       & 1082    & 1078    & 1091     & 1093               & 1089      \\
N                                              & 376      & 376        & 376     & 376     & 376      & 376                & 376       \\
Adjusted-R2                                    & 0.00     & 0.02       & 0.03    & 0.09    & 0.06     & 0.07               & 0.13      \\
Condition Number                               & 1.00     & 1.00       & 2.05    & nan     & nan      & nan                & nan       \\
\hline
\end{tabular}
\end{center}
\end{table*}
\begin{table*}
\caption{Severity AND}
\begin{center}
\begin{tabular}{lccccccc}
\hline
                                & Activity &			 Nontrivial & Satoshi & Network & Weighted & Network*Nontrivial &   All    \\
\hline
\hline
\end{tabular}
\begin{tabular}{llllllll}
Intercept                                      & 0.00     & 0.00       & 0.00    & 0.00    & 0.00     & 0.00               & 0.00     \\
                                               & (0.00)   & (0.00)     & (0.00)  & (0.00)  & (0.00)   & (0.00)             & (0.00)   \\
nontrivial                                     &          &            &         &         & 0.00     &                    &          \\
                                               &          &            &         &         & (0.00)   &                    &          \\
user1 betweenness centrality weighted          &          &            &         &         &          &                    & 0.00     \\
                                               &          &            &         &         &          &                    & (0.00)   \\
user1 closeness centrality unweighted          &          &            &         & 0.00    &          & 0.00               & 0.15**   \\
                                               &          &            &         & (0.00)  &          & (0.00)             & (0.06)   \\
user1 closeness centrality weighted            &          &            &         &         & 0.24***  &                    & 0.00     \\
                                               &          &            &         &         & (0.05)   &                    & (0.00)   \\
user1 closeness centrality weighted:nontrivial &          &            &         &         &          & 0.00               &          \\
                                               &          &            &         &         &          & (0.00)             &          \\
user1 clustering coefficient                   &          &            &         & 0.00    &          & 0.00               & 0.23***  \\
                                               &          &            &         & (0.00)  &          & (0.00)             & (0.05)   \\
user1 clustering coefficient:nontrivial        &          &            &         &         &          & 0.00               &          \\
                                               &          &            &         &         &          & (0.00)             &          \\
user1 days since first post                    &          &            &         &         & 0.00     &                    & -0.01    \\
                                               &          &            &         &         & (0.00)   &                    & (0.06)   \\
user1 degree incoming                          &          &            &         & 0.00    & 0.00     & 0.00               & -0.09    \\
                                               &          &            &         & (0.00)  & (0.00)   & (0.00)             & (0.07)   \\
user1 degree outgoing                          &          &            &         &         &          &                    & 0.00     \\
                                               &          &            &         &         &          &                    & (0.00)   \\
user1 degree total                             &          &            &         &         & 0.00     &                    & 0.00     \\
                                               &          &            &         &         & (0.00)   &                    & (0.00)   \\
user1 num posts                                &          &            &         &         & 0.00     &                    & -0.08    \\
                                               &          &            &         &         & (0.00)   &                    & (0.07)   \\
user1 num subjects                             &          &            &         &         &          &                    & -0.01    \\
                                               &          &            &         &         &          &                    & (0.06)   \\
user1 pagerank weighted                        &          &            &         &         & 0.14***  &                    & 0.55***  \\
                                               &          &            &         &         & (0.05)   &                    & (0.18)   \\
user1 satoshi distance                         &          &            &         &         &          &                    & -0.11    \\
                                               &          &            &         &         &          &                    & (0.11)   \\
user1 satoshi distance inf                     &          &            &         &         &          &                    & 0.08     \\
                                               &          &            &         &         &          &                    & (0.10)   \\
user1 satoshi pagerank weighted                &          &            &         &         &          &                    & -0.22    \\
                                               &          &            &         &         &          &                    & (0.17)   \\
R2                                             & 0.00     & 0.00       & 0.00    & 0.00    & 0.18     & 0.00               & 0.27     \\
ElasticNet CV MSE:                             & 1.01     & 1.01       & 1.01    & 0.94    & 0.92     & 1.04               & 0.90     \\
BIC                                            & 1072     & 1072       & 1072    & 1090    & 1040     & 1102               & 1030     \\
N                                              & 376      & 376        & 376     & 376     & 376      & 376                & 376      \\
Adjusted-R2                                    & 0.00     & 0.00       & 0.00    & -0.01   & 0.16     & -0.01              & 0.25     \\
Condition Number                               & 1.00     & 1.00       & 1.00    & 1.80    & 13.45    & 2.30               & nan      \\
\hline
\end{tabular}
\end{center}
\end{table*}

%\section{Linear model coeficients}
%\balancecolumns

% That's all folks!
\end{document}


