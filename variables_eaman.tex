\subsection{Network Structure}
In this section, we discuss the various metrics extracted from discussion networks and used as independent variables in our regression analysis. Many of these variables are standard metrics in graph theory designed to capture node centrality in specific scenarios \cite{KleinbergNetworks}. As mentioned before, each coin is associated with a forum user and a discussion network which corresponds to the state of the forum at the time the user introduced the coin to the community. All of our node-level variables refer to the user introducing the coin. Below, we list the network variables included in the analysis. We used Python igraph implementation for computing the network-related metrics \cite{igraph}.
\begin{enumerate}[topsep=0pt,itemsep=-0.5ex,partopsep=1ex,parsep=1ex]
  \item \textbf{Introducer number of posts:} The total number of posts (thread-initiations or simple replies) the coin introducer has made at the time she introduces the coin. This captures the user's level of activity in the community.
  \item \textbf{Introducer number of threads:} The total number of threads the coin introducer has made. Users who start many threads are more likely to receive incoming edges and to shape the dialogue in the community.
  \item \textbf{Seniority:} Is the number of days since the user's first post in the forums. We use this as a proxy for the user's seniority in the community.
  \item \textbf{Incoming degree:} The (incoming) degree centrality captures the role of dialogue-shapers in the community as it is the number of unique users who have replied to any of the focal user's threads.
  \item \textbf{Outgoing degree:} The (outgoing) degree centrality captures the role followers in the community as it is the number of unique thread initiators the focal user has ever replied to.
  \item \textbf{Total degree:} The (undirected) degree centrality captures the total level of the user's involvement in the community in any of the two forms above.
  \item \textbf{Clustering Coefficient:} A measure of embeddedness or triadic closure, this is the fraction of the focal user's triads that are closed. It does not use the direction of the edges and measures how `tightly-knit' are the connections between the user and others who have ever engaged with her either by replying to her thread or receiving a reply from her. The triadic closure is closely related to the principle of balance which states that if two user pairs A-B and B-C are connected, the existence of a tie between A and C on the triad further strengthens it and removes any potential strain that could exist between the A-B and B-C relation. In general, ideas are more likely to be reinforced and persistent in a triad if it is closed. Such a positive effect of balance or triadic closure on tie qualities and their persistence is shown to exist in online social network such as Twitter \cite{KleinbergBalance}, and we believe the same argument applies to our scenario.
  TODO: DO WE NEED A BETTER INTERPRETATION HERE FOR TRIADIC CLOSURE?
  %JULIAN: To be honest I'd do away with most of the above definitions for the social networks track, and just quickly list the features. A quick explanation of why they ought to be useful would be valuable, but definitions for these basic concepts is overkill.
  \item \textbf{Unweighted closeness centrality:} While degree centrality measures the level of user engagement in the community, it only examines the local structure around the user. In other words, it's possible for the focal user to have a high degree centrality, but only in an isolated subcommunity. Closeness centrality, while related to degree centrality, measures the level of the user's engagement with the global network \textit{either directly or indirectly}. It is relevant in many scenarios, including in online discussions, as information spreads via shortest paths. Closeness centrality for the i\super{th} user is defined as:
  \begin{equation}
    C_{i} = \sum_{j=1}^{N} \frac{N-1}{|S_{ij}|}
  \end{equation}
  where $S_{ij}$ denotes the set of users on the shortest path from i to j. It is a normalized sum of distances from the focal user (i) to all other users (j), where edges all are weighted with a distance of 1. The sum of all inverse distances is normalized by the number of users present in the network at the time of coin introduction, so that the comparison between the closeness centrality of various users (who introduce the coins) at different times  is valid. 
  
  In our context, a user with high incoming closeness centrality has initiated many threads and received replies from a diverse set of users who themselves are close to a large set of diverse users. Similarly, a user with high outgoing closeness centrality has replied to a diverse set of thread-initiators who themselves are close to a large set of diverse users. Our analysis consisted of three versions of the unweighted closeness centrality:
  \begin{enumerate}
    \item \textbf{Incoming:} Only the directed paths leading to the focal user are used. In other words, it measures closeness of the whole network to the focal user. Users who start many threads are likely to have higher incoming closeness centrality.
    \item \textbf{Outgoing:} Only the directed paths starting from the focal user to all the other users are used. In other words, it measures closeness of the focal user to the whole network. Users who reply to many threads are likely to have higher outgoing closeness centrality.
    \item \textbf{Undirected:} The paths both from and to the focal users are used. Users who initiate and reply to many threads are likely to have higher undirected closeness centrality.
  \end{list}
  \item \textbf{Weighted closeness centrality:} Similar to the unweighted closeness centrality above, we computed three different versions, with the exception that edges are weighted to indicate the distance or level of interaction intensity between the two users. The edges weights in our discussion network are determined by the frequency of interactions between two users; and as two users interact more, they are deemed to be closer in their shortest path. Thus in the computation of weighted closeness centralities, we use the reciprocal of the weights as the distance between two users.
  \begin{equation}
    C_{i} = (N-1)\sum_{j=1}^{N}\sum_{e \in S_{ij}} w_{e}
  \end{equation}
  where $e$ denotes an edge in $S_{ij}$ the set of users on the shortest path from i to j. $w_e$ is the weight of edge $e$ determined by the number of interactions between the end points.
  
  \item \textbf{betweenness centrality weighted:}
  \item \textbf{satoshi distance:}
  \item \textbf{satoshi pagerank weighted:}
  \item \textbf{pagerank weighted:}
\end{enumerate}