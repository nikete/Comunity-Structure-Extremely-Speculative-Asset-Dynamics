Our results suggest that bubble dynamics are be strongly influenced by timing effects. Traditional network measures based on introduciers position in the aggregate discussion graph provide some aditional  information beyond this effect.

As future work we are looking at coarse (positive/negative, detailed/cursory) semantic analysis of the discussion, and evidence of prior cooperation between pairs of participants in other altcoin markets, in order to attempt a more accurate characterization of core participants and their actions.

The features of the node as well as the construction of the graph are informed by a the pre-existing literature which does not allow for much complexity on the edges of the graphs beyond weight and direction.
A promising avenue for future work is to explore richer models for edges, in particular allowing them to exist in latent spaces, where he influence or attnetion that is payed by responding to a user is mediated by the wording of the response. 
The cross-sectional design with time separation does not allow us to take advantage of intra-coin variation.
Beyond using data from forums, code repositories could also be exploited in future work as a rich source of variation beyond the binary feature of whether a coin is a trivial fork or not.

While the period of active speculation in these alternative cryptocurrencies seems to be largely over, further speculative markets seem likely to emerge in the future. Comparative analysis over mutliple social structures of speculative bubbles that have disagregated infromation could allow for much more refined tests of theories. 
All data and code in this paper is public (including tools for scrapping it and disargegated raw price and forum data), and substantial effort has been made to docuement it, so it can be used in such future investigations. 
