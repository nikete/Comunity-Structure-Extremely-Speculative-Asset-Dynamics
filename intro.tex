\section{Introduction}

Speculative bubbles, since at least the dutch tulip mania (\cite{garber2001famous} pp 127-31 for references) periodically take over markets. %not quite what he says; fundamentals driven then morals story.
The public notoriety of Bitcoin and the massive price increases that it has seen relative to its starting prices a few years ago have lead to an explosion of attempts to create \''the next bitcoin\'' often referred to as cryptocurrencies or \''coins\''.
While speculation during bubbles as a social process that has been theoretically studied \cite{abolafia1988enacting, earl2007decision, bakker2010social, harras2011grow}, data on a real social network underpinning it has not been used.
The attempts to introduce these new \''coins\'' largely take place in an online forum called Bitcoin talk.
We present a novel dataset based on the Bitcoin Talk forum that allows us to identify the introducers of each coin and build measures of their position in the network based on which users have engaged with whom in the forum before the coin is announced or traded (TODO clarify on when which).
By considering the community structure that exists in the forum before a coin is introduced we are able to sidestep problems of reverse causation that would plague a analysis that relied on in time variation between prices and network structure. 
We also collect the set of prices and traded volumes across the cryptocurrencies that are introduced in the discussions on the forums, and construct measures of both the intensity (aka severity, for each dollar invested at the peak what could be recovered on average) and the magnitude (how many dollars or bitcoins where nominally traded in the asset).
This allows us to evaluate the predictive power of the features of the node in our constructed network that corresponds to the user who first introduces the coin. 
While the magnitude of the assets traded is small relative to most financial and commodity markets, it is much larger than even the most lavishly funded experimenter could hope for.
Furthermore, the rich market structure that surrounds (some 45 exchanges appear on the dataset, ranging in credibility from VC backed and registered in the US, to anonymous and mysteriously run) provides a rich source of institutional variation with extremely open data, a striking contrast to most financial or commodities market trade level data. 
Our contribution aims to begin in the computational social sciences a field that would have a place in the sociology of markets analogous to that of computational imaging lesion studies do in neuropsychology \footnote{ \cite[cosma2008]}.

The evidence based uncovered by “— traces of their communicative interactions as they work out their thoughts about matters of common concern” 

\being{quote}
"Bitcoin is a purely online virtual currency, unbacked by either physical commodities or sovereign obligation; instead, it relies on a combination of cryptographic protection and a peer-to-peer protocol for witnessing settlements. Consequently, Bitcoin has the unintuitive property that while the ownership of money is implicitly anonymous, its flow is globally visible. In this paper we explore this unique characteristic further, using heuristic clustering to group Bitcoin wallets based on evidence of shared authority, and then using re-identification attacks (i.e., empirical purchasing of goods and services) to classify the operators of those clusters. From this analysis, we characterize longitudinal changes in the Bitcoin market, the stresses these changes are placing on the system, and the challenges for those seeking to use Bitcoin for criminal or fraudulent purposes at scale." 
\end{quote} 
\cite{meiklejohn2013fistful}

While states can create the demand required for a currency system to run by compelling tax payment in it (for a recent example), non state sponsored currencies must find some other ways of creating demand.
The initial market for which bitcoin has been used (prices denominated in it, transactions only in it) was drug sales.\footnote{A overview of the different drug maretplaces and estimated transaction volumes can be found in \cite{soska2015measuring}. To the best of the authors knowledge no other sector beyond speculation has even remotely subsntantial volume at present; a very primitive form of unregulated gambling Satoshi Dice, did for a brief poitnin thepast)   }
Since the cost of producing a new coin is effectively zero, new currencies have thus been floated with every single drug name possible. Many chains can claim to the same name, so exchanges with volume (since speculation is the only possible use of almost all of the coins) become de-facto arbitrators of who has a minimally viable claim.\footnote{While it is theretically possibl to engage in a distributed protocol to exchange between two cryptocurrencies, see part II of lecture 10 in \cite{princeton10}}

 “rapid rise of options” with the rise of the theory to price those options, in having currencies with zero entry costs
MacKenzie, Donald. "An Engine, Not a Camera."






%Methodologically the free parameters in the way we do the weights on the weighted graph  is horrible, a millon free parameters get introduced. follow up work for another paper: do some unsupervised feature learning over the dam forums threads to build the network; use some internal validity metric . Find political way of saying this in future work.

