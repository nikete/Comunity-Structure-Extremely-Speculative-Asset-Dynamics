\section{Introduction}

Speculative bubbles , since at least the dutch tulip mania (Shiller 2005) periodically take over markets.
While speculation during bubbles as a social process that has been theoretically studied \cite{abolafia1988enacting, earl2007decision, bakker2010social} data on the social network has not been used in such studies. 
The public notoriety of Bitcoin and the massive price increases that it has seen relative to its starting prices a few eyars ago have lead to an explosion of attempts to create \''the next bitcoin\''.
The attempts to introduce these new \''coins\'' largely take place in a forum called Bitcoin talk, we present a novel dataset constructed on this data that allows us to identify the introducers of each coin 

https://books.google.com.au/books?id=0TsEM40mepQC&pg=PA144&lpg=PA144&dq=network+structure+and+bubbles&source=bl&ots=3F0pQ7LRPY&sig=JDO3QCEVqJX7KRN0h8Qltm7KgVQ&hl=en&sa=X&ved=0CCwQ6AEwAmoVChMIt_Tz1LHByAIVhqGUCh0x6AoA#v=onepage&q=network%20structure%20and%20bubbles&f=false

The explosion of cryptocurrencies that follows, in their online comunities and 
and the dynamics of the markets around it provide a chance to observe highly detailed data on a cross ection of comparable asset bubbles.
We collect and make public a dataset of the bitcointalk discussion forum (posts, authors) and use to construct a graph of whcih users have engaged with whom in the forum. 
We also collect a  set of prices and traded volumes accross the cryptocurrencies that are itnroduced in the disucssions on the forums.
This allows us to evaluate the predictive power 

A simple 

Provide time separation argument

Provide 


We study the power of structural features of the social network around cryptocurrencies to understand the severity of bubbles that occur in them. 
Our goal is to asses the predictive capacity of structural featres of derived from the social networks that completely preceed tradingin the asset.
Bubble forecasting in economics and finance has used features of the time series
 All our measures are constructed using observations to contruct a network before the relevant cryptocurrency is ever traded. 



“ it would have a place in the sociology of markets analogous to that of lesion studies in neuropsychology.” a window into the social life of the mind https://www.aaai.org/Papers/Symposia/Spring/2008/SS-08-06/SS08-06-017.pdf

The evidence based uncovered by “— traces of their communicative interactions as they work out their thoughts about matters of common concern” 

crash of 87 http://www.iijournals.com/doi/abs/10.3905/jpm.1989.409242?journalCode=jpm “rapid rise of options” witht he rise of the thoery to price those options, in having currencies with zero entrie costs
MacKenzie, Donald. "An Engine, Not a Camera."
While states can create the demand required for a currency system to run by compelling tax payment in it (for a recent example), non state sponsored currencies must find some other wyas of creating demand.
The initial market for which bitcoin has been used (prices denominated in it, transactions only in it) where drug sales. citation.
New currencies have thus been floated with every single drug name posible. Many chains can claim to the same claim, so exchanges (since speculation is the only possible use of allmost all of them) become de-facto stablishers of who has a minimaly viable claim. 


"We develop a series of cross-sectional regression specifications to forecast skewness in the daily returns of individual stocks. Negative skewness is most pronounced in stocks that have experienced (1) an increase in trading volume relative to trend over the prior six months, consistent with the model of Hong and Stein (NBER Working Paper, 1999), and (2) positive returns over the prior 36 months, which fits with a number of theories, most notably Blanchard and Watson's (Crises in Economic and Financial Structure. Lexington Books, Lexington, MA, 1982, pp. 295–315) rendition of stock-price bubbles. Analogous results also obtain when we attempt to forecast the skewness of the aggregate stock market, though our statistical power in this case is limited."






Data for the historical valuations of the coins was scrapped from coinmarketcap.com The bubble measure was constructed as the log of the ratio in the loss of value from the maximum notional valuation of the available currency supply to to the present value. 

Our network measurement data consisted of discussions in Bitcoin and Altcoin online forums on bitcointalk.org from January 2010 until July 2014. 
Bitcoin forum is mainly concerned with Bitcoin and related technological issues whereas Altcoin discussions are concerned about alternative cryptocurrencies. Each forum consists of hundreds of threads and each thread contains many posts.  Given the forum discussion data at the level of individual posts, we constructed the undirected graph of user discussions. In this network, nodes are the forum users and are connected by an edge if the corresponding users have ever interacted in any thread. The edge weights between two users are determined based on their interaction frequency. Furthermore, the weights are adjusted by size of the thread (i.e. an interaction in small threads counts more than an interaction in a large thread) and a decay factor (i.e. a recent interaction counts more than an old interaction). We constructed the network over time by replaying the posts and updating the discussion graph accordingly.  Whenever a new digital currency was introduced in the forum for the first time, a snapshot of the graph was taken and used for extracting various network measures corresponding to the coin. It is important to analyze the discussion graph before the coin is mentioned for the first time to avoid any possible confounding factors. The network measures we used were as followed: number of edges, number of nodes, density, diameter, clustering coefficient, average path length and average degree. We also computed the following network measures for the user who introduced the crypto coin for the first time: clustering coefficient, closeness centrality, betweenness centrality and laplacian centrality. As future work, we will construct the directed “Satoshi” network at time of the coin introduction where an edge exists from user a to user b, if user a has ever replied to user b. Relevant features of this network such as “Satoshi” distance of the coin introducer and her eigenvector centrality can be used in analysis of bubbles.
A OLS fit of bubble severity to the network structure measures fails to provide statistically significant explanatory power. Inspection of the model fit reveals that the model does however successfully estimates high bubble values for the most severe bubbles, and there is never a severe bubble when it predicts low severity.


Tâtonnement stability of infinite horizon models with saddle-point instability
WP Heller - Econometrica: Journal of the Econometric Society, 1975 - JSTOR
... horizon case. 5 As Samuelson [17, p. 981] notes, ". . . history tells us that all tulip
manias have ended in finite time.... Every bubble is someday pricked." Knowing this,
people start to sell when prices reach ridiculous heights. But at ...





