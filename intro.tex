Speculative bubbles are perceived to periodically take over markets \cite{garber2001famous}.
Going back at least to the \emph{South Sea} bubble in the early 18\textsuperscript{th} century, well-informed parties  have invested knowingly in bubbles, and found it profitable \cite{temin2004riding}.
Today, the public notoriety of Bitcoin, together with its massive price increases and their associated publicity lead to an explosion of attempts to create ``the next Bitcoin''.
Collectively, these currencies are often referred to as ``cryptocurrencies'' or ``cryptocoins'' or simply ``coins'', and a vibrant set of exchanges have emerged where these are traded, either for each other or money.
The vast majority of these coins have no possible value, and their markets would appear to be driven largely by speculation.
Many of them appear to be nothing but attempts at turning a quick profit from inflating the implied valuation of a coin shortly after creating it.
This is driven by the extremely low cost and effort required to create a new coin, with most being minimal changes to parameters and branding of a pre-existing codebase.

Those who make and trade these coins communicate largely online, and much of their activity is concentrated on public forums.
Moreover, price and volume data from the exchanges is freely available and widely aggregated\footnote{Exchanges are however largely unregulated and often anonymously owned. There is no way to account how much of reported volumes are manipulation attempts by exchange insiders. The existence of attempts at arbitrage between them places some bounds on how blatantly the data can be manipulated for prices, but volumes are impossible to assess. }, and public source code to all cryptocoins is the  norm.
This makes cryptocoins an ideal lens through which to study the social life of a ``market mania'' \cite{cosma2008}, and the role of social structure in economic outcomes\cite{Granovetter-outcomes}.
Such a study is valuable both as a means of understanding the dynamics of bubbles themselves, but also from a computational social science perspective, to understand the ecosystem and lifecycles of online communities around highly speculative technologies and markets more broadly.
Such a study can serve in the computational social sciences a role analogous to that of lesion studies do in neuropsychology: by studying disfunctional marketplaces we can better understand the underlyingmechanisms that may not be exposed in more functional ones.
While extnesive literature studies the comunities around other online marketplaces is emerging \cite{sastry2015predicting,gilbert2013need,luckman2013aura,chang2014specialization} we belive this is the first attempt to study the social structure of these cryptocurrencies.


We present a study of a large number of cryptocurrencies.
We compile a a novel dataset that combines measures derived from social networks of users interacting in online cryptocurrency forums, market data aggregated over dozens of exchanges, and properties of the software implementing the cryptocurrencies themselves.
From forums, we identify the users who introduce each coin to the community and build measures of their structural position in the network based on
their engagement patterns in forum threads \emph{before} the coin is ever traded (usually within a few days after their announcement on the forum).
In this way, we identify 626 coins that are announced by users of the forum and can be mapped to price and volume data from the exchanges.
Next, from price and transaction volume data, we build measures of speculation that quantify the size of the bubble resulting from trading activity on each coin. 
Many of the new coins announced on the forums experience a frenzied period of rising prices as users are looking for the next Bitcoin. However in many of these cases, prices sharply fall as soon as the users discover the shortcomings in the design, learn about the lack of real innovation or simply because first adapters observe about a weak community response to the new coin. For this reason, the cryptocurrency ecosystem presents a great opportunity for studying the mechanisms behind formation of bubbles, due to the abundance of such cases.

%\begin{figure}
%\includegraphics[width=\columnwidth]{AuroraCoin}
%\end{figure}

\begin{figure}
\centering
\includegraphics[width=0.45\columnwidth]{severity-btc.pdf}
\includegraphics[width=0.45\columnwidth]{magnitude-btc.pdf}
\\
\includegraphics[width=0.45\columnwidth]{severity-usd.pdf}
\includegraphics[width=0.45\columnwidth]{magnitude-usd.pdf}
\caption{Severity and magnitude both decrease as a function of the time the coin is introduced to the market.}
\label{severity_boxplot}
\end{figure}

While the mechanisms that drive bubbles have been studied both theoretically 
\cite{abolafia1988enacting,earl2007decision,bakker2010social,harras2011grow}
and experimentally
\cite{moinas2013bubble},
% in the lab, 
studying the social networks of those promoting the asset has not been previously possible due to the lack of an exhaustive dataset. 
To the best of our knowledge, our work is the first to analyze the role that the global structural features of the communication network, such as the position of certain stakeholders, can play in promoting the asset.
While the magnitude of the assets traded in our study is small relative to most financial and commodity markets, it is nevertheless much larger than any plausible experiment in an academic environment.
For example, the largest bubble in our dataset, AuroraCoin, reached a valuation of one billion USD in March 2014,
with reported daily trading volumes of 6.8M USD. It shed 90\% and 99\% of its value in a week and well under a year, respectively.
To provide some context, this is equivalent to one quarter of Iceland's entire foreign exchange reserves in 2014 \footnote{4.1 Billon USD, The World Bank, Global Economic Monitor, accessed October 2015}, the population of which AuroraCoin promoters claimed they would distribute half of the coins to.

Using price and volume data, we construct measures of both the \textbf{magnitude} and the \textbf{severity} of bubbles.
These are defined formally in our variable section %\ref{variables_nikete},
but their intuition is that we say the magnitude of a bubble is large when a high volume of trades measured in dollars happens,
% TODO CLARIFY OUR USD and BTC MEASURES
while a coin has a severe bubble if investing a fixed amount leads to losing a large proportion of that amount. 
%By considering the community structure that exists in the forums before a coin is traded, we are able to predict a substantial fraction of the variation in the severity and a smaller fraction of the magnitude of the resulting bubble.
%This is a challenging task as models that rely on either simple activity or network metrics show almost no predictive out-of-sample power, and are unable to explain even 1\% of the variation in either tasks,
% while our best model performs an order of magnitude better in both tasks and does not  use the forum activity information after trading on each coin starts. 
%JULIAN: R^2 of 0.1 (if I interpret correctly) doesn't sound impressive until you've argued the fundamental difficulty of the problem, you might wait to mention such numbers.
%JULIAN: Would sound more impressive if you just said "it performs an order of magnitude better"

%The main driver of our explanatory power is the centrality of users in the directed communication network derived from the forum. As it can be seen in Figure \ref{severity_boxplot}, the severity of the bubbles increase with the centrality of the users who introduce new coins to the community. However, both the severity and the magnitude of bubbles decrease with the seniority of such users.


%EAMAN: commented out. repetion of what is said below.
%This effect appears to be mediated by whether a coin involves a nontrivial technological innovation, the direction of the interaction reversing depending on whether it relates to magnitude or severity.

Interestingly this effect is concentrated in different ways depending on whether the coin software is more than a trivial modification to existing code base: as shown in Figure \ref{volume_severity_centrality} trivial coins have more severe bubbles the more central their introducers are, while volume is greater the more central the introducer of a nontrivial coin is.

%Methodologically the free parameters in the way we do the weights on the weighted graph  is horrible, a million free parameters get introduced. follow up work for another paper: do some unsupervised feature learning over the dam forums threads to build the network; use some internal validity metric . Find political way of saying this in future work.

