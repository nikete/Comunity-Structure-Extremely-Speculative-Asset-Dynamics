\section{Data Description}



\subsection{Prices and Exchanges}
TODO: NIKETE I did not remove this, I just moved it down to analysis vars. Wanted this subsection
to only talk about the data and not how you make the bubble measure.

Our main outcome measures are the severity of the inflation an asset price, and the magnitude of money transacted in it.
We operationalize the intensity of a bubble as the proportion of a 1 dollar (TODO check currency base) that would be lost buying at the maximum price and selling after that proportionally to the volume of the market till the present, we call this severity.
We define the volume as the sum of the contemporaneous dollar (todo check) volume of trade.
As a secondary outcome measure we consider the number of exchanges that list the coin.
We scraped two major price aggregators; coinmarketcap and coininfo.


\subsection{Forum Discussions}
NOTE: EAMAN WROTE THIS

In order to study the effect of communication network around cryptocoins on
price variations, we collected all the posts from the most popular cryptocurrency
online community, bitcointalk.  Our data consisted of all the posts that were
made between January 2010 and July 2015 on the most active crypto-related forums:
\begin{enumerate}
  \item{Bitcoin Discussion:} This is the oldest forum on the website which mainly focuses
    on issues only related to Bitcoin. Interestingly, Satoshi Nakamoto, the alleged
    creator of Bitcoin made the first post on this forum in January 2010 and
    was active until January 2011. The presence of Satoshi in the data set enables us
    to study the position of various actors in the online community relative to Satoshi
    and its relation with the success or failure of cryptocoins they advocate or reject.

  \item{Altcoin Discussion:} This is the most active forum in the community
    with more than 730,000 posts as of July 2015, and dating back to June 2011.
    The discussions in this forum mainly evolve around alternative currencies
    other than Bitcoin. Users often discuss the merits or flaws of various
    altcoins or simply exchange technical information.
  
  \item{Announcement (Altcoin):} Community announcements such as development of 
    exchange client or addition of new features are made here. This is an important forum
    in our study as the creation of new altcoins are announced here. Whenever a new
    altcoin is announced to the community, the announcement is tagged with string ANN.
    This enables us to detect announcement of new coins into the market and identify
    the users who introduced them for the first time.

  \item{Mining (Altcoin):} Technical issues pertaining to mining (i.e. validating transactions)
    altcoins are discussed here.
  \item{Marketplace (Altcoin):} This forum contains the discussions on a wide-range of 
    market-related issues, such as price or volume trends, possible pump and dump schemes
    and exchange tips.

\end{enumerate}
TODO: CHECK AVERAGE NUMBER OF POSTS PER THREAD, and other statistics

Each forum consists of many subjects or threads initiated by different users.
Each thread contains several posts or replies, with an average of 10 posts per
thread.  The reply structure within each thread constitutes the basis of our
forum network, discussed below.  Each post has several fields which contain
valuable information in our context.
\begin{enumerate}
  \item{Subject:} Usually, the initiator of the thread chooses subject and all the
    following posts inherit the same subject.
  \item{Content:} The actual text of the message.
  \item{Position in the thread}: The later posts in the thread might not be as important
    as earlier posts and could be about issues other than the original topic of the thread.
  \item{Author}
  \item{Date}
\end{enumerate}

The community had only 10000 unique users until early 2013, however it grew
considerably faster after 2013 and reached about 70000 by early
2015. Nevertheless, there are only 10000 active users within any 30 day period on average.


\section{Forum Interaction Network}
NOTE: EAMAN WROTE THIS

Given the forum discussion data at the level of individual posts, we can construct a
network capturing the structure of user discussions.  In this network, nodes
are the forum users and edges represent some form of interaction between the
users.

We experimented with different network construction methods, weight assignment
schemes, and edge retention periods. For example, edges could be added based on
co-appearance in each thread. However, this approach quickly leads to a very
dense network revealing little information about discussion dynamics. Moreover,
the network could be unweighted or weights could be assigned to each edge based
on a frequency of interactions between the two endpoints. Weights could
also adjusted by size of the thread (i.e. an interaction in small threads counts
more than an interaction in a large thread) and a decay factor (i.e. a recent
interaction counts more than an old interaction).
Finally, the network could contain the interactions only within the past few
days to capture short-term effect of information diffusion and influence
dynamics. Alternatively, network construction could use all the interactions
since the inception of the forums. The unlimited retention of interactions
leads to a better view on long-term dynamics and more accurate identification
of influential and senior users in the forum.

In the following discussion, we only include the results from directed unlimited network.
The network is directed since the \textit{edges point from posters within each thread to thread-initiators}. The omission
of co-appearance edges leads to a sparser network which isolates the communication
patterns around ``dialogue-shapers''. The network is unlimited since all such interactions
from the inception of bitcointalk are included in its construction. The longer timespan
of interactions captures relevant information on seniority and community influence 
which are obtained through long-term and persistent presence in the forums.
We edges on this network are weighted by the number of times a poster replies
to a thread-initiator in unique threads. We chose the unlimited directed network
since it proved to be the most informative and simplest to interpret.

Prior to construction of the network, we merged posts from all forums into a
single large large forum since the community base of all five forums mentioned
above is made of the same users and we are mostly concerned about influence and
aggregate information flow among users, rather than the exact topic of the discussion.
The network construction involves replaying the posts over time sorted by their date and updating the
discussion graph accordingly. Whenever a new altcoin was introduced
in the forum for the first time, the user who introduced it and a snapshot of the graph was taken.
The position of the first introducer in the network snapshot and its general structure 
will be used for extracting various network measures corresponding to that coin. Our analysis
uses these per-coin for evaluating the performance of each coin.

TODO: NIKETE, in the following paragraph, please EXPLAIN BETTER WHY WE CHOSE THE FIRST INTRODUCTION TIME.

In order to avoid any possible confounding factors, it is important to analyze
the discussion network at the time of coin introduction to the community. The
identification of true introductions of new altcoins is a difficult process
prone to many false-positives. Fortunately, majority of such introductions are
made in the \textit{Announcement} forum and are preceded with the ``ANN'' tag.
Our identification strategy looked for mentions of a new altcoin only in the
first post of a thread, as new coins are usually introduced to the community in
the form of an announcement.  In particular, we look for the first mention of
both the coin symbol \textbf{and} its descriptive name in the subject of a
thread which also contains the announcement tag. The first mentions of either
the coin symbol \textbf{or} the its name are used fall-back in case the more
restrictive \textbf{and} requirement did not detect the coin.  Using this
identification strategy, we were able to detect the first introduction of 554 altcoins
out of 679. The forum user who initiated such a thread is assigned as the 
introducer of the coin to the community.


\section{Analysis Variables}
\subsection{Prices and Exchanges}
Our main outcome measures are the severity of the inflation an asset price, and the magnitude of money transacted in it.
We operationalize the intensity of a bubble as the proportion of a 1 dollar (TODO check currency base) that would be lost buying at the maximum price and selling after that proportionally to the volume of the market till the present, we call this severity.
We define the volume as the sum of the contemporaneous dollar (todo check) volume of trade.
As a secondary outcome measure we consider the number of exchanges that list the coin.


\subsection{Network Structure}
TODO: EAMAN WILL WRITE THIS


We computed the following network measures for the users who introduced the
crypto coin for the first time to the forum:
\begin{enumerate}
  \item{num mentions}
  \item{num posts}
  \item{num subjects}
  \item{days since first post}
  \item{degree total}
  \item{degree incoming}
  \item{degree outgoing}
  \item{clustering coefficient}
  \item{closeness centrality unweighted}
  \item{closeness centrality weighted}
  \item{closeness centrality incoming unweighted}
  \item{closeness centrality outgoing unweighted}
  \item{closeness centrality incoming weighted}
  \item{closeness centrality outgoing weighted}
  \item{betweenness centrality weighted}
  \item{satoshi distance}
  \item{satoshi pagerank weighted}
  \item{pagerank weighted}
\end{enumerate}
