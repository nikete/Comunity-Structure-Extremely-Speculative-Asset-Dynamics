
\section{Analysis Variables}
\subsection{Prices and Exchanges}
Our main outcome measures are the severity of the inflation an asset price, and the magnitude of money transacted in it.
We operationalize the intensity of a bubble as the proportion of a 1 dollar (TODO check currency base) that would be lost buying at the maximum price and selling after that proportionally to the volume of the market till the present, we call this severity.
We define the volume as the sum of the contemporaneous dollar (todo check) volume of trade.
As a secondary outcome measure we consider the number of exchanges that list the coin.


\subsection{Network Structure}
In this section, we discuss the various metrics extracted from discussion networks and used as independent variables in the regression analysis. Many of these variables are standard metrics in graph theory designed to capture node centrality is specific scenarios \cite{KleinbergNetworks}. As mentioned before, each coin is associated with a forum user and a discussion network which corresponds to the state of the forum at the time the user introduced the coin to the community. All of our node-level variables refer to the user introducing the coin. Below, we list the network variables included in the analysis. We used Python igraph implementation for computing the network-related metrics \cite{igraph}.
\begin{enumerate}[topsep=0pt,itemsep=-0.5ex,partopsep=1ex,parsep=1ex]
  \item{Introducer number of posts:} The total number of posts (thread-initiations or simple replies) the coin introducer has made at the time she introduces the coin. It captures the user's level of activity in the community.
  \item{Introducer number of threads:} The total number of threads the coin introducer has made. Users who start many threads are more likely to receive incoming edges and to shape the dialogue in the community.
  \item{Seniority:} It is the number of days since the user's first post in the forums. We use this as a proxy for user's seniority in the community.
  \item{Incoming degree:} The (incoming) degree centrality captures the role of dialogue-shapers in the community as it is the number of unique users who have replied to any of the focal user's threads.
  \item{Outgoing degree:} The (outgoing) degree centrality captures the role followers in the community as it is the number of unique thread initiators the focal user has ever replied to.
  \item{Total degree:} The (undirected) degree centrality captures total level of user's involvement in the community in any of the two forms above.
  \item{Clustering Coefficient:} A measure embeddedness or triadic closure, it is the fraction of focal user's triads that are closed. It does not use the direction on the edges and measures how tightly knit the focal user is connected to the other users who have ever engaged with her either by replying to her thread or receiving a reply from her. The triadic closure is closely related to the the principle of balance which states that if two user pairs A-B and B-C are connected, the existence of a tie between A and C on the triad further strenghtens it and removes any potential strain that could exist between A-B and B-C relation. In general, ideas are more likely to be reinforced and persistent in a triad if it is closed. Such a positive effect of balance or triadic closure on tie qualities and their persistence is shown to exist in online social network such as Twitter \cite{KleinbergBalance}, and we believe the same argument applies to this scenario.
  TODO: DO WE NEED A BETTER INTERPRETATION HERE FOR TRIADIC CLOSURE?
  \item{closeness centrality unweighted:} While degree centrality measures the level of user engagement in the community, it only examines the local structure around nodes. closeness  is the sum of distances to all other nodes. identify the nodes which could reach others quickly. degree centrality 
  \item{closeness centrality weighted:}
  \item{closeness centrality incoming unweighted:}
  \item{closeness centrality outgoing unweighted:}
  \item{closeness centrality incoming weighted:}
  \item{closeness centrality outgoing weighted:}
  \item{betweenness centrality weighted:}
  \item{satoshi distance:}
  \item{satoshi pagerank weighted:}
  \item{pagerank weighted:}
\end{enumerate}
